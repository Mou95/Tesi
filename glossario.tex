
%**************************************************************
% Acronimi
%**************************************************************
\renewcommand{\acronymname}{Acronimi e abbreviazioni}

%**************************************************************
% Glossario
%**************************************************************
%\renewcommand{\glossaryname}{Glossario}

\newglossaryentry{nlp}
{
    name=\glslink{nlp}{NLP},
    text=Natural Language Processing,
    sort=nlp,
    description={in informatica con il termine \emph{Application Programming Interface API} (ing. interfaccia di programmazione di un'applicazione) si indica ogni insieme di procedure disponibili al programmatore, di solito raggruppate a formare un set di strumenti specifici per l'espletamento di un determinato compito all'interno di un certo programma. La finalità è ottenere un'astrazione, di solito tra l'hardware e il programmatore o tra software a basso e quello ad alto livello semplificando così il lavoro di programmazione}
}



\newglossaryentry{chatbot}
{
    name=\glslink{chatbot}{Chatbot},
    text=Chatbot,
    description={in informatica con il termine \emph{Chatbot} si indica un programma che simula una conversazione tra un robot e un essere umano. Questi programmi funzionano o come utenti stessi delle \emph{chat} o come persone che rispondono alle FAQ delle persone che accedono al sito}
}

\newglossaryentry{FAQ}
{
    name=\glslink{faq}{FAQ},
    text=Frequently Asked Question,
    description={in informatica con il termine \emph{Chatbot} si indica un programma che simula una conversazione tra un robot e un essere umano. Questi programmi funzionano o come utenti stessi delle \emph{chat} o come persone che rispondono alle FAQ delle persone che accedono al sito}
}
