% !TEX encoding = UTF-8
% !TEX TS-program = pdflatex
% !TEX root = ../tesi.tex

%**************************************************************
\chapter{Conclusioni}
\label{cap:conclusioni}

\section{Raggiungimento degli obiettivi}
Come descritto nella sezione \ref{obiettivi}, io e il mio tutor aziendale abbiamo stilato una serie di obiettivi da raggiungere nelle 300 ore di stage, divisi in obiettivi obbligatori e desiderabili. \\ 
Gli obiettivi obbligatori si concentravano sull'analisi di mercato dei principali strumenti per il \gls{NLP} e sullo sviluppo delle funzionalità del prodotto, integrandole nell'architettura già creata dall'azienda. Gli obiettivi desiderabili si focalizzavano invece nel testare in modo approfondito il risultato ottenuto, attraverso l'utilizzo dei \gls{chatbot} di Facebook Messenger in un ambiente locale. 
Al termine dello stage tutti gli obiettivi prefissati sono stati soddisfatti nei tempi previsti. Questo è stato possibile grazie alla buona pianificazione delle tempistiche necessarie allo svolgimento dei vari compiti, sia da parte dello stagista che da parte dell'azienda. La metodologia di sviluppo \emph{Agile} si è inoltre dimostrata molto efficace, permettendomi di rispondere in modo rapido ai cambiamenti proposti dall'azienda durante il mio lavoro. 
\section{Conoscenze acquisite}
Da un punto di vista formativo l'attività di stage è stata sicuramente molto positiva. Queste 300 ore mi hanno permesso di mettermi alla prova, dandomi un riscontro su quanto gli anni universitari mi hanno preparato, sia nell'ambito delle mie conoscenze, sia da quello umano, per affrontare il mondo del lavoro.


Lo stage ha sicuramente arricchito il mio bagaglio personale di competenze tecniche, dandomi una panoramica delle procedure e delle attività che giornalmente si svolgono all'interno di un'azienda. Questo mi ha inoltre permesso di rendermi conto quanto una buona collaborazione tra i membri di un team sia fondamentale nello sviluppo di un prodotto, di qualsiasi tipo si tratti.
Le competenze acquisite non riguardano solamente il campo informatico, grazie alle nuove tecnologie utilizzate, ma anche le mie capacità organizzative, fondamentali per il rispetto dei vincoli e delle scadenze imposte dall'azienda. Ho dovuto infatti pianificare in modo preciso il lavoro delle mie settimane, per garantire ad \azienda{} la portata a termine del mio progetto, senza dover impiegare un dipendente per adempire alle mie mancanze. 

\section{Valutazione personale}
Nel complesso ritengo la mia esperienza di stage molto positiva ed istruttiva. L'inserimento, se pur breve, in un contesto aziendale permette di far capire ad uno studente quanto siano differenti il mondo del lavoro e quello universitario, in modo da cogliere gli aspetti fondamentali che solamente esperienze di questo tipo ti possono dare.

Per quanto riguarda l'azienda che mi ha ospitato si è rivelata molto disponibile ed accogliente, mi ha guidato nei miei primi giorni di stage chiarendomi tutti i dubbi o le mie mancanze dal punto di vista tecnico. Il progetto che mi è stato proposto mi ha permesso di confrontarmi con nuove tecnologie, mettendo alla prova la mia capacità di apprendere nuovi concetti in breve tempo. Mi sono inoltre reso conto di quanto le mie esperienze universitarie, soprattutto per quanto riguarda i progetti individuali e di gruppo svolti, si sono dimostrate fondamentali per organizzare e gestire una quantità di lavoro che mai avevo dovuto affrontare.

Mi ritengo quindi molto soddisfatto dell'esperienza fatta, in un campo come l'informatica dove apprendere nuove nozioni e competenze resta un aspetto fondamentale per crescere e migliorare. 