% !TEX encoding = UTF-8
% !TEX TS-program = pdflatex
% !TEX root = ../tesi.tex

%**************************************************************
\chapter{Progettazione}
\label{cap:progettazione}

La progettazione nel caso del prodotto che dovevo sviluppare si è concentrata particolarmente sulla piattaforma di api.ai, in quanto rappresentava la maggior parte del lavoro. Successivamente sono passato alla progettazione delle classi da introdurre nel codice già creato dall'azienda, per gestire le nuove funzionalità del \gls{chatbot}.

\section{Studio di api.ai}
Prima di passare all'attività di progettazione è stato fondamentale analizzare e studiare a fondo le possibilità che api.ai mette a disposizione per la creazione del prodotto a me richiesto. I concetti base per capire il funzionamento di api.ai sono quattro:
\begin{itemize}
	\item \textbf{agent}: gli \emph{agents} sono meglio descritti come moduli NLU (Natural Language Understanding). Questi possono essere inclusi nell'applicazione, nel prodotto o nel servizio e trasformano le richieste di utenti naturali in dati attivi. Questa trasformazione si verifica quando un input utente corrisponde a uno degli \emph{intent} all'interno dell'\emph{agent};
	\item \textbf{intent}: sono una mappatura tra quello che l'utente può scrivere in input e l'azione che il software deve intraprendere. Un intent è formato dalle seguenti sezioni:
	\begin{itemize}
		\item\textbf{ \emph{user says}}: perché l'\emph{agent} capisca la domanda, sono necessari esempi di come la stessa domanda può essere posta in modi diversi. Lo sviluppatore aggiunge queste permutazioni alla sezione \emph{user says} dell'\emph{intent}. Più variazioni vengono aggiunte all'\emph{intent}, meglio l'\emph{agent} comprenderà l'utente;
		\item \textbf{\emph{action}}: contiene il nome della \emph{action}, che può essere utilizzato per attivare una particolare funzione del prodotto, e la tabella dei \textbf{\emph{parameters}}. I parameters possono gli elementi che collegano le parole nelle \emph{user says} alle \emph{entities};
		\item \textbf{\emph{response}}: in questa sezione è possibile definire la risposta di api.ai quando l'\emph{intent} viene attivato. Non è stato quasi mai utilizzato, in quanto la risposta all'utente veniva generata nella \emph{business logic}.
	\end{itemize}
	\item \textbf{context}: i \emph{context} rappresentano il contesto corrente della richiesta di un utente. Ciò è utile per differenziare frasi che possono essere vaghe o avere significati diversi a seconda delle preferenze dell'utente, della posizione geografica, della pagina corrente di un'applicazione o dell'argomento della conversazione. È possibile impostare un \emph{lifespan} ad ognuno di essi per definire dopo quante richieste il \emph{context} deve scadere;
	\item \textbf{entity}: le entities sono strumenti potenti utilizzati per estrarre i valori dei parametri dagli input degli utenti. Tutti i dati importanti che si desidera ottenere dalla richiesta di un utente, avranno un'entità corrispondente. Le \emph{system entities} sono entità pre-costruite fornite da API.AI per facilitare la gestione dei più comuni concetti (luoghi, orari, colori, ecc..). È possibile poi definire le proprie \emph{entities} in base alle necessità dello sviluppatore;
\end{itemize}


\section{Progettazione agents api.ai}
Durante la progettazione degli \emph{agents} per api.ai è stato necessario definire tutti gli \emph{intents} utili a soddisfare i requisiti definiti durante l'analisi dei requisiti. Il passo successivo è stato quello di progettare le \emph{user says} per ogni \emph{intent} e le relative \emph{entity}.

\subsubsection{Gestore di eventi}
Per quanto riguarda la progettazione del \gls{chatbot} dedicato alla gestione di eventi, gli \emph{intents} che mi sono serviti per soddisfare tutti i requisiti sono stati i seguenti:
\begin{itemize}
	\item \textbf{durata\_conferenza}: permette all'utente di domandare la durata di una conferenza e viene attivato con domande come: \emph{"Quanto dura la conferenza Y?"}. La risposta del \gls{chatbot} contiene il nome, l'inizio, la fine e la durata (in minuti o in ore) della conferenza richiesta dall'ospite;
	\begin{figure}[h]
		\centering
		\includegraphics[scale=0.12]{../Immagini/meteo_scelta.png}
		\caption{dfsf}
	\end{figure}
	\item \textbf{luogo\_conferenza}: permette all'utente di domandare il luogo dove si svolgerà la conferenza e viene attivato con domande come: \emph{"Dove si svolge la conferenza Y?"}. La risposta del \gls{chatbot} contiene un carosello predefinito da Messenger, con tutte le informazioni sull'aula in questione;
	\item \textbf{ora\_conferenza}: permette all'utente di l'orario di inizio e di fine di una conferenza e viene attivato con domande come: \emph{"A che ore inizia la conferenza Y?"}. La risposta del \gls{chatbot} contiene un carosello predefinito da Messenger, con tutte le informazioni sulla conferenza;
	\item \textbf{indicazioni\_stanza}: permette all'utente di domandare le indicazioni per trovare una determinata aula e viene attivato con domande come: \emph{"Dammi delle indicazioni per la stanza X"}. La risposta del \gls{chatbot} contiene le indicazioni presenti nel database, con una piccola mappa illustrativa;
	\item \textbf{programma\_giornata}: permette all'utente di domandare il il programma dell'evento di un determinato giorno e viene attivato con domande come: \emph{"Qual è il programma di oggi?"}. La risposta del \gls{chatbot} contiene un carosello per ogni conferenza in programma quel giorno;
	\item \textbf{programma\_no\_data}:
	\item \textbf{data\_ora\_stanza\_conferenza}:
	\item \textbf{data\_stanza\_conferenza}:
	\item \textbf{ora\_stanza\_conferenza}:
	\item \textbf{visualizza\_agenda}:
	\item \textbf{richiesta\_aiuto}:
\end{itemize}

\subsection{Dati ARPA Veneto}
Il \gls{chatbot} dedicato al meteo è stato realizzato grazie agli \emph{open data} messi a disposizione dall'Agenzia Regionale per la Prevenzione e Protezione Ambientale del Veneto(ARPAV). Ogni giorno, nel sito ufficiale\footcite{arpav}, vengono emessi tre bollettini:
\begin{itemize}
	\item \textbf{alle 9:00}: che rappresenta un aggiornamento del bollettino del giorno precedente;
	\item \textbf{alle 13:00}: il nuovo bollettino;
	\item \textbf{alle 16:00}: un aggiornamento del bollettino emesso alle 13.
\end{itemize} 

Il file XML che è possibile scaricare, contiene queste informazioni:
\begin{itemize}
	\item le previsioni dei cinque giorni successivi per le 18 zone in cui è stata divisa la regione del Veneto;
	\item una descrizione dell'evoluzione generale dei cinque giorni successivi, per tre macro zone: la regione intera, la zona delle Dolomiti e la pianura veneta.
\end{itemize}

Ad ogni nuova emissione del bollettino, i nuovi dati vengono inseriti nel database aziendale, in modo da comunicare agli utenti solamente le notizie più aggiornate.

\subsubsection{Meteo Veneto Bot}

Gli intents che ho deciso di creare per soddisfare tutti i requisiti sono i seguenti:
\begin{itemize}
	\item \textbf{richiesta\_meteo}: permette all'utente di chiedere le previsioni del meteo specificando una giornata o un periodo di tempo (es. weekend) e il comune di interesse (se non viene specificato, si considera il comune da lui selezionato all'inizio dell'interazione con il \gls{chatbot}). La risposta contiene un carosello con il meteo richiesto.
	\begin{figure}[h!]
		\centering
		\includegraphics[scale=0.12]{../Immagini/richiesta_meteo.png}
		\caption{Esempio di }
	\end{figure}	
	\item \textbf{richiesta\_sole}: permette all'utente di chiedere se è previsto il sole in una specifica giornata o un periodo di tempo (es. weekend), in un determinato comune. La risposta è formata da due messaggi: il primo mostra le giornate dove è previsto il sole, tra quelle richieste dall'utente, il secondo contiene i caroselli delle previsioni.
	\begin{figure}[h!]
		\centering
		\includegraphics[scale=0.12]{../Immagini/richiesta_sole.png}
		\caption{Esempio di }
	\end{figure}	
	\item \textbf{richiesta\_pioggia}: permette all'utente di chiedere se è prevista pioggia in una specifica giornata o un periodo di tempo (es. weekend), in un determinato comune. La risposta è formata da due messaggi: il primo mostra le giornate dove è prevista pioggia, tra quelle richieste dall'utente, il secondo contiene i caroselli delle previsioni.
	\begin{figure}[h!]
		\centering
		\includegraphics[scale=0.12]{../Immagini/richiesta_pioggia.png}% "%" necessario
		\qquad\qquad
		\includegraphics[scale=0.12]{../Immagini/richiesta_pioggia2.png}
		\caption{Didascalia comune alle due figure}
	\end{figure}
	\item \textbf{richiesta\_nebbia}: permette all'utente di chiedere se è prevista nebbia in una specifica giornata o un periodo di tempo (es. weekend), in un determinato comune. La risposta è formata da due messaggi: il primo mostra le giornate dove è prevista nebbia, tra quelle richieste dall'utente, il secondo contiene i caroselli delle previsioni.
	\begin{figure}[h!]
		\centering
		\includegraphics[scale=0.12]{../Immagini/richiesta_nebbia.png}
		\caption{Esempio di }
	\end{figure}
	\item \textbf{richiesta\_neve}: permette all'utente di chiedere se è prevista neve in una specifica giornata o un periodo di tempo (es. weekend), in un determinato comune. La risposta è formata da due messaggi: il primo mostra le giornate dove è prevista neve, tra quelle richieste dall'utente, il secondo contiene i caroselli delle previsioni.
	\begin{figure}[h!]
		\centering
		\includegraphics[scale=0.12]{../Immagini/richiesta_nebbia.png}
		\caption{Esempio di }
	\end{figure}
	\item \textbf{richiesta\_bel\_tempo}: permette all'utente di chiedere se è previsto bel tempo in una specifica giornata o un periodo di tempo (es. weekend), in un determinato comune. La risposta è formata da due messaggi: il primo mostra le giornate dove è previsto bel tempo, tra quelle richieste dall'utente, il secondo contiene i caroselli delle previsioni.
	\begin{figure}[h!]
		\centering
		\includegraphics[scale=0.12]{../Immagini/richiesta_bel_tempo.png}
		\caption{Esempio di }
	\end{figure}
	\item \textbf{richiesta\_brutto\_tempo}: permette all'utente di chiedere se è previsto brutto tempo in una specifica giornata o un periodo di tempo (es. weekend), in un determinato comune. La risposta è formata da due messaggi: il primo mostra le giornate dove è previsto brutto tempo, tra quelle richieste dall'utente, il secondo contiene i caroselli delle previsioni.
	\begin{figure}[h!]
		\centering
		\includegraphics[scale=0.12]{../Immagini/richiesta_brutto_tempo.png}
		\caption{Esempio di }
	\end{figure}
	\item \textbf{richiesta\_temperature}: permette all'utente di chiedere le temperature previste in una specifica giornata o un periodo di tempo (es. weekend), in un determinato comune. La risposta contiene le temperature massimi e minime previste fornite da ARPA Veneto.
	\begin{figure}[h!]
		\centering
		\includegraphics[scale=0.12]{../Immagini/richiesta_brutto_tempo.png}
		\caption{Esempio di }
	\end{figure}
	\item \textbf{ascolta\_bollettino}: permette all'utente di chiedere il bollettino audio emesso da ARPA Veneto ogni giorno. La risposta contiene il file audio richiesto.
	\begin{figure}[h!]
		\centering
		\includegraphics[scale=0.12]{../Immagini/bollettino.png}
		\caption{Esempio di }
	\end{figure}
	\item \textbf{fenomeni\_particolari}: permette all'utente di chiedere se sono presenti avvisi o fenomeni particolari emessi da ARPAV. La risposta contiene questi avvisi, se presenti.
	\begin{figure}[h!]
		\centering
		\includegraphics[scale=0.12]{../Immagini/bollettino.png}
		\caption{Esempio di }
	\end{figure}

\end{itemize}


\section{Jaro Winkler distance}
Durante l'attività di progettazione mi sono reso conto come fosse necessario gestire un possibile errore di scrittura dell'utente in una delle sue domande, soprattutto nelle parole fondamentali per formulare le risposte, come ad esempio il nome di un comune per il \gls{chatbot} del meteo o il nome di una conferenza in quello degli eventi. In un primo momento infatti l'input dell'utente veniva utilizzato direttamente nelle \emph{query SQL} per interrogare il database ed ottenere i dati di interesse, soprattutto attraverso l'operatore \emph{"LIKE"}. In questo modo però non è possibile gestire il caso in cui un utente scriva ad esempio il comune "Padvoa", intendendo Padova. \\
Per ovviare a questo problema quindi è stato deciso di introdurre, dopo uno studio delle possibili soluzioni, la Jaro Winkler distance\footcite{jaro}, ossia una metrica che misura la "distanza" tra due stringhe per capire quanto esse siano simili tra loro. Grazie a questa accortezza, nel caso di errore di scrittura, il \gls{chatbot} è in grado di:
\begin{itemize}
	\item fornire una serie di opzioni di cosa secondo lui l'utente voleva scrivere, dando la possibilità ad esso di selezionare quella giusta;
	\item fornire i dati richiesti dall'utente nel caso ci sia un'unica corrispondenza simile a quanto scritto dall'utente nel database.
\end{itemize}