% !TEX encoding = UTF-8
% !TEX TS-program = pdflatex
% !TEX root = ../tesi.tex

%**************************************************************
\chapter{Verifica e validazione}
\label{cap:verifica}

Durante il mio stage lo sviluppo delle nuove funzionalità ha portato alla minimizzazione dei tempi di verifica e validazione. Tale decisione è stata presa in comune accordo con il tutor aziendale, in quanto lo scopo dello stage mirava all’estensione di più funzionalità possibili, che potranno essere testate successivamente dall’azienda.
Questa decisione è inoltre appoggiata dalla metodologia \emph{agile} utilizzata nello sviluppo del progetto, la quale definisce la qualità del software come la capacità di soddisfare i bisogni del cliente piuttosto che soddisfare metriche fissata a priori.
In ogni caso, vista l'importanza di queste attività, sono state adottate tecniche di analisi statica e analisi dinamica per il codice sorgente, al fine di verificarne e validarne il comportamento.

\section{Analisi statica}
L’analisi statica è il processo di valutazione di un sistema o di un suo componente basato sulla sua forma, struttura, contenuto, documentazione senza che esso sia eseguito. Gli strumenti di analisi statica del codice consentono di individuare porzioni di codice del proprio programma ad alta probabilità di errore. Avendo a disposizione una lista di linee di codice sospette, un programmatore può poi verificare se siano presenti errori e, in caso positivo, rivedere il codice corrispondente e correggere le problematiche individuate. \\
In particolare sono state utilizzate queste tecniche di analisi statica:
\begin{itemize}
	\item \textbf{\emph{code inspections}} o \textbf{\emph{reviews}}, in cui si effettua una lettura ed analisi di gruppo
con l’ausilio della documentazione. Qui si analizza il codice secondo delle \emph{checklist}, ossia liste contenenti dei tipici errori di programmazione, indipendenti dal linguaggio usato e dal codice in esame;	
	\item \emph{walktrough\textbf{•}}, un’analisi dinamica con casi di test svolti a mano.
\end{itemize}

\section{Analisi dinamica}
L’analisi dinamica è il processo di valutazione di un sistema software o di un suo componente basato sulla osservazione del suo comportamento in esecuzione. Una volta soddisfatti tutti gli obiettivi definiti con l’azienda, l’attività di validazione è stata svolta tramite analisi dinamica grazie alla progettazione di alcuni test. I test sviluppati sono di due tipi:
\begin{itemize}
	\item \textbf{ciao}
\end{itemize}

\section{Verifica e validazione api.ai}
Gran parte del progetto di stage verteva nell'istruire gli \emph{agent} di api.ai per gestire le domande degli utenti fatte attraverso i \gls{chatbot}. Essendo quindi una parte molto importante del prodotto sono stati creati alcuni test per verificare e validare le sue funzionalità. 