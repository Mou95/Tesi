% !TEX encoding = UTF-8
% !TEX TS-program = pdflatex
% !TEX root = ../tesi.tex

%**************************************************************
\chapter{Il contesto aziendale}
\label{cap:contesto-aziendale}
%**************************************************************

Introduzione al contesto applicativo.\\

\noindent Esempio di utilizzo di un termine nel glossario \\
%\gls{api}. \\

\noindent Esempio di citazione in linea \\
\cite{site:agile-manifesto}. \\

\noindent Esempio di citazione nel pie' di pagina \\
citazione\footcite{womak:lean-thinking} \\


\begin{description}
    \item[{\hyperref[cap:introduzione]{Il secondo capitolo}}] descrive ...
    
    \item[{\hyperref[cap:descrizione-stage]{Il terzo capitolo}}] approfondisce ...
    
    \item[{\hyperref[cap:analisi-requisiti]{Il quarto capitolo}}] approfondisce ...
    
    \item[{\hyperref[cap:progettazione-codifica]{Il quinto capitolo}}] approfondisce ...
    
    \item[{\hyperref[cap:verifica-validazione]{Il sesto capitolo}}] approfondisce ...
    
    \item[{\hyperref[cap:conclusioni]{Nel settimo capitolo}}] descrive ...
\end{description}

Riguardo la stesura del testo, relativamente al documento sono state adottate le seguenti convenzioni tipografiche:
\begin{itemize}
	\item gli acronimi, le abbreviazioni e i termini ambigui o di uso non comune menzionati vengono definiti nel glossario, situato alla fine del presente documento;
	\item per la prima occorrenza dei termini riportati 
	nel glossario viene utilizzata la seguente nomenclatura: \emph{parola}\glsfirstoccur{};
	\item i termini in lingua straniera o facenti parti del gergo tecnico sono evidenziati con il carattere \emph{corsivo}.
\end{itemize}

\section{Profilo aziendale}
\azienda{} è un'azienda che nasce nel 2003 a Belluno, dove tuttora mantiene la propria sede. Nei primi anni si dedica allo sviluppo di applicativo web per conto di terzi, lavorando in stretta collaborazione con aziende di comunicazione per consulenze di tipo tecnico. Parallelamente inizia la creazione del prodotto di punta dell'azienda: SMSHosting. Si tratta di un \emph{gateway} per l'invio e la ricezione di sms professionali da web, che consente agli utenti di comunicare con i propri clienti direttamente dalla propria area riservata, oppure dall'esterno tramite email , FTP, moduli web, software Windows, app per smartphone e molto altro.  \\
Nel corso degli anni, e con l'introduzione di nuove tecnologie, l'azienda evolve i propri prodotti, introducendo nuovi strumenti di comunicazione per i propri clienti, focalizzando le proprie competenze sui mobile che dal 2009 iniziavano a diffondersi.
Viene così creata una piattaforma dedicata all'invio di messaggi PUSH tramite le più diffuse \emph{messaging apps}, come Facebook Messenger e Telegram. I clienti possono utilizzare questo servizio, integrato nella piattaforma SMSHosting, attraverso email o delle semplici API REST.\\
Come ultima novità \azienda{} ha iniziato lo sviluppo di \gls{chatbot} per Facebook Messenger e Telegram. L'azienda mette a disposizione sia una piattaforma semplice e intuitiva dove un utente può creare il proprio \gls{chatbot} in pochi passi, seguendo i più comuni template di business, sia la possibilità di crearne di nuovi secondo le richieste del cliente.

\section{Dominio applicativo}

\section{Tecnologie utilizzate}
Le tecnologie principali che vengono utilizzate da \azienda{} per lo sviluppo dei propri prodotti possono essere divise in tre diverse aree:
\begin{itemize}
	\item applicazioni iOS e Android native;
	\item applicazioni web con tecnologie Java;
	\item frontend.
\end{itemize} 

\subsection{Applicazioni iOS e Android native}
Le tecnologie utilizzate per lo sviluppo di applicazioni mobile dipende naturalmente dal sistema operativo dove si vuole sviluppare. L'azienda utilizza anche dei \emph{framework cross-platform} come \emph{PhoneGap}.
\subsubsection{Android}
Per quanto riguarda \emph{Android} l'azienda si affida al linguaggio nativo di questo sistema operativo, cioè \emph{Java}. 
Il team di sviluppo ha grande conoscenza di questo linguaggio, agevolando così lo sviluppo dei nuovi prodotti.
\subsubsection{iOS}
Per lo sviluppo di applicazioni \emph{iOS} il linguaggio che viene principalmente utilizzato è \emph{Objective-C}, un linguaggio ben noto a coloro che devono codificare questa versione dei prodotti.

\subsection{Applicazioni web con Java}
Le applicazioni web su piattaforma Java sono nel DNA di \azienda{} fin dalla nascita nel 2003. \\
Il team di sviluppo ha grande conoscenza dei principali \emph{framework} di sviluppo ed ha lavorato a progetti di grande dimensione utilizzando sia CMS open source (Open CMS) che commerciali (Broadvision, Vignette OpenText).
Il \emph{framework} Spring, Hibernate ORM, Quartz Scheduler, Jersey for RESTful Web services sono solo pochi esempi delle librerie comunemente utilizzate e che fanno parte del \emph{core} dei loro prodotti.

\subsection{Frontend}
Nello sviluppo della parte \emph{frontend} dei propri portali e applicativi web, \azienda{} mira ad utilizzare gli strumenti più innovativi per garantire la massima velocità di presentazione e la compatibilità con i \emph{device} di nuova generazione. \\
I principali linguaggi e \emph{framework} sono HTML5, CSS3, JQuery, Bootstrap e Sencha.

\section{Processi aziendali}
\subsection{Metodologia}
Fino dai suoi inizi \azienda{} crea delle soluzioni software altamente dipendenti dalle specifiche dei propri clienti. Per potere fare ciò è indispensabile mantenere una stretta comunicazione con il cliente, per capire le sue volontà e le sue richieste in modo preciso. La realizzazioni di nuovi progetti deve quindi essere in grado di reagire al cambiamento dei requisiti anche in fase di sviluppo, in quanto il cliente può cambiare idee e giudizi sulle funzionalità de proprio prodotto.\\
La metodologia di sviluppo più adatta, adottata quasi completamente dall'azienda, è la Adaptive Software Development.
Il metodo agile utilizzato da \azienda{} prevede una forte e frequente collaborazione con il cliente, in modo da ricevere dei \emph{feedback} puntuali sugli incrementi portati al prodotto. In ogni momento il cliente si trova ad avere un prodotto via via più completo, che rispecchia i requisiti da esso imposti e discussi con l'azienda.\\
La chiave per il successo di questa metodologia è racchiusa in questi punti:
\begin{itemize}
	\item sviluppare qualcosa di utile;
	\item coltivare la fiducia degli \emph{stakeholders};
	\item costituire gruppi di lavoro competenti e collaborativi;
	\item far sì che il team abbia la possibilità e sia in grado di prendere decisioni;
	\item consegnare spesso nuove versioni all'aggiunta di nuove funzionalità;
	\item incoraggiare l'adattabilità;
	\item cercare di ottenere l'eccellenza tecnica;
	\item quando possibile, aumentare il volume di dati immessi.
\end{itemize}

\subsection{Strumenti a supporto dei processi}
\subsubsection{Gestione di progetto}
Per la gestione di progetto \azienda{} utilizza Asana per l'assegnazione di task relativi a nuove attività da svolgere, \emph{bug} da correggere o manutenzione da svolgere e BeeBole come \emph{web timesheet}.
\paragraph{Asana}
\subsubsection{Gestione di versione}
\subsubsection{Comunicazione aziendale}


