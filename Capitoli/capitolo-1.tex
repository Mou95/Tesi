% !TEX encoding = UTF-8
% !TEX TS-program = pdflatex
% !TEX root = ../tesi.tex

%**************************************************************
\chapter{Il contesto aziendale}
\label{cap:contesto-aziendale}
%**************************************************************

\section{Profilo aziendale}
\azienda{} è un'azienda che nasce nel 2003 a Belluno, dove tuttora mantiene la propria sede. Nei primi anni si dedica allo sviluppo di applicativo web per conto di terzi, lavorando in stretta collaborazione con aziende di comunicazione per consulenze di tipo tecnico. Parallelamente inizia la creazione del prodotto di punta dell'azienda: \textbf{SMSHosting}\footcite{smshosting}. Si tratta di un \emph{gateway} per l'invio e la ricezione di sms professionali da web, che consente agli utenti di comunicare con i propri clienti direttamente dalla propria area riservata, oppure dall'esterno tramite email , \gls{FTP}, moduli web, software Windows, app per \emph{smartphone} e molto altro.  \\
Nel corso degli anni, e con l'introduzione di nuove tecnologie, l'azienda evolve i propri prodotti, introducendo nuovi strumenti di comunicazione per i propri clienti, focalizzando le proprie competenze sui mobile che dal 2009 iniziavano a diffondersi.
Viene così creata una piattaforma dedicata all'invio di \textbf{messaggi \gls{push}} tramite le più diffuse \emph{messaging apps}, come Facebook Messenger e Telegram. I clienti possono utilizzare questo servizio, integrato nella piattaforma SMSHosting, attraverso email o delle semplici \gls{API} REST.\\
Come ultima novità \azienda{} ha iniziato lo sviluppo di \gls{chatbot} per Facebook Messenger e Telegram. L'azienda mette a disposizione sia una piattaforma semplice e intuitiva dove un utente può creare il proprio \gls{chatbot} in pochi passi, seguendo i più comuni template di business, sia la possibilità di crearne di nuovi secondo le richieste del cliente.

\section{Tecnologie utilizzate}
Le tecnologie principali che vengono utilizzate da \azienda{} per lo sviluppo dei propri prodotti possono essere divise in tre diverse aree:
\begin{itemize}
	\item applicazioni \emph{iOS} e \emph{Android} native;
	\item applicazioni web con tecnologie \emph{Java};
	\item \gls{frontend}.
\end{itemize} 

\subsection{Applicazioni \emph{iOS} e \emph{Android} native}
Le tecnologie utilizzate per lo sviluppo di applicazioni mobile dipende naturalmente dal sistema operativo dove si vuole sviluppare. L'azienda utilizza anche dei \glspl{framework} \emph{cross-platform} come \emph{PhoneGap}.
\subsubsection{Android}
Per quanto riguarda \emph{Android} l'azienda si affida al linguaggio nativo di questo sistema operativo, cioè \emph{Java}. 
Il team di sviluppo ha grande conoscenza di questo linguaggio, agevolando così lo sviluppo dei nuovi prodotti.
\subsubsection{iOS}
Per lo sviluppo di applicazioni \emph{iOS} il linguaggio che viene principalmente utilizzato è \emph{Objective-C}, un linguaggio ben noto a coloro che devono codificare questa versione dei prodotti.

\subsection{Applicazioni web con Java}
Le applicazioni web su piattaforma \emph{Java} sono nel DNA di \azienda{} fin dalla nascita nel 2003. \\
Il team di sviluppo ha grande conoscenza dei principali \gls{framework} di sviluppo ed ha lavorato a progetti di grande dimensione utilizzando sia CMS \gls{open source} (Open CMS) che commerciali (Broadvision, Vignette OpenText).
Il \gls{framework} \emph{Spring}, \emph{Hibernate} \gls{ORM}, \emph{Quartz Scheduler},\emph{ Jersey for RESTful Web services} sono solo pochi esempi delle librerie comunemente utilizzate e che fanno parte del \emph{core} dei loro prodotti.

\subsection{Frontend}
Nello sviluppo della parte \gls{frontend} dei propri portali e applicativi web, \azienda{} mira ad utilizzare gli strumenti più innovativi per garantire la massima velocità di presentazione e la compatibilità con i \emph{device} di nuova generazione. \\
I principali linguaggi e \glspl{framework} sono \emph{HTML5, CSS3, JQuery, Bootstrap e Sencha}.

\section{Processi aziendali}
\subsection{Metodologia}
Fino dai suoi inizi \azienda{} crea delle soluzioni software altamente dipendenti dalle specifiche dei propri clienti. Per potere fare ciò è indispensabile mantenere una stretta comunicazione con il cliente, per capire le sue volontà e le sue richieste in modo preciso. La realizzazioni di nuovi progetti deve quindi essere in grado di reagire al cambiamento dei requisiti anche in fase di sviluppo, in quanto il cliente può cambiare idee e giudizi sulle funzionalità de proprio prodotto.\\
La metodologia di sviluppo più adatta, adottata quasi completamente dall'azienda, è la \gls{ASD}.
Il metodo agile utilizzato da \azienda{} prevede una forte e frequente collaborazione con il cliente, in modo da ricevere dei \emph{feedback} puntuali sugli incrementi portati al prodotto. In ogni momento il cliente si trova ad avere un prodotto via via più completo, che rispecchia i requisiti da esso imposti e discussi con l'azienda.\\
La chiave per il successo di questa metodologia è racchiusa in questi punti:
\begin{itemize}
	\item sviluppare qualcosa di utile;
	\item coltivare la fiducia degli \glspl{stakeholder};
	\item costituire gruppi di lavoro competenti e collaborativi;
	\item far sì che il team abbia la possibilità e sia in grado di prendere decisioni;
	\item consegnare spesso nuove versioni all'aggiunta di nuove funzionalità;
	\item incoraggiare l'adattabilità;
	\item cercare di ottenere l'eccellenza tecnica;
	\item quando possibile, aumentare il volume di dati immessi.
\end{itemize}

\subsection{Strumenti a supporto dei processi}
\label{Teconologie}
\subsubsection{Gestione di progetto}
Per la \textbf{gestione di progetto} \azienda{} utilizza Asana\footcite{asana} per l'assegnazione di \emph{task} relativi a nuove attività da svolgere o \emph{bug} da correggere e BeeBole\footcite{beebole} come \emph{web timesheet}.
\paragraph{Asana}
Asana è un \gls{SaaS} \emph{web based} che mira a migliorare la collaborazione all'interno dei team di lavoro. Permette infatti la gestione di progetti e \emph{tasks} online, senza dover utilizzare le email.\\
\azienda{} ha creato un proprio \emph{workspace}, dove poter aggiungere nuovi progetti e nuovi \emph{tasks} assegnati a questi progetti. \\ \\
Per ogni \emph{task} che si vuole creare, è possibile specificare una serie di dettagli:
\begin{itemize}
	\item \textbf{nome e descrizione} relativa;
	\item il \textbf{membro del team} che ha il compito di svolgere quel task. La persona deve essere registrata all'interno del \emph{workspace} di \azienda;
	\item la \textbf{data} entro la quale il compito deve essere svolto. Una settimana prima l'incaricato riceverà una mail di promemoria;
	\item uno o più \textbf{\emph{tags}} per differenziare i diversi \emph{tasks} all'interno del progetto.
\end{itemize}
Una volta assegnato il \emph{task}, incaricato e assegnatario, più tutte le persone che sono in grado di visualizzarlo, potranno aprire una conversazione dedicata nella sezione specifica di quel \emph{task}, dove scambiarsi idee, file e molto altro. Terminato il lavoro, l'incaricato dovrà marcare il \emph{task} come concluso attraverso l'apposita spunta.

\paragraph{BeeBole}
BeeBole è uno strumento di \textbf{\emph{web timesheet}} che dà la possibilità ai propri utenti di monitorare in modo efficiente il tempo dedicato a progetti, clienti e incarichi. \\
L'azienda \azienda{} sfrutta questo applicativo per controllare il budget dedicato ad ogni progetto e le ore effettivamente investite da ogni componente del team, per procedere con una fatturazione corretta e trasparente nei confronti del cliente.
\subsubsection{Gestione di versione}
L'azienda \azienda{} utilizza come \textbf{sistema di controllo di versione} del codice \gls{Mercurial}. In particolare, per
poter raccogliere tutto il codice derivante dai vari progetti delle varie aree sviluppo, viene utilizzata la versione premium di BitBucket\footcite{bit}.\\

Alcuni vantaggi dell'adozione di un sistema di versionamento del codice sono:
\begin{itemize}
	\item \textbf{ridondanza}: ogni sviluppatore possiede un \emph{backup} della \emph{repository} localmente, limitando così la possibilità di perdita totale dei dati;
	\item \textbf{disponibilità}: anche in assenza di connessione alla \emph{repository} principale, è possibile continuare ad effettuare \emph{commit} ed a lavorare. Una volta ripristinata
la connessione, la \emph{repository} locale può essere sincronizzata con quella remota
rendendo le modifiche visibili a tutti;
	\item \textbf{\emph{branch e merge}}: permette con molta facilità la creazione dei cosiddetti \emph{branch}, ossia delle ramificazioni dal prodotto stabile presente nella \emph{repository}, per apportare modifiche o nuove funzionalità evitando di introdurre errori e bug. Una volta che anche questa nuova \emph{feature} risulta corretta, può essere inserita all'interno del progetto principale tramite un \emph{merge}.
\end{itemize}
\subsubsection{Comunicazione aziendale}
Come strumento di \textbf{comunicazione} tra i dipendenti dell'azienda, \azienda{} ha scelto Slack\footcite{slack}.
Slack è un software che rientra nella categoria degli strumenti di collaborazione aziendale utilizzato per inviare messaggi in modo istantaneo ai membri del team. Il suo punto di forza è la possibilità di creare dei \textbf{canali} dedicati a un progetto o ad un particolare argomento, dove vengono inseriti tutti o solo parte dei dipendenti aziendali. È possibile inoltre comunicare con il team anche attraverso chat individuali o chat con due o più membri.\\
La scelta è ricaduta su questo strumento anche per la sua facilità di utilizzo e la caratteristica di essere fruibile da tutti i dispositivi \emph{iOs, Android, Windows} come applicazione e da \emph{web browser}.

\section{Clienti}
Grazie ai prodotti offerti da \azienda{} la sua clientela spazia in molti ambiti diversi come automobilismo, banche, telecomunicazioni e molto altro. Queste aziende si rivolgono a \azienda{} sia per migliorare le comunicazione tra loro e la propria clientela, attraverso la piattaforma di SMSHosting e le sue numerose soluzioni, sia per proporre lo sviluppo di nuovi prodotti \emph{web based}. \\
Il team di \azienda{} ha lavorato, direttamente o tramite i loro partner, con clienti in tutta Italia. 
Grazie ad una consolidata modalità di lavoro riescono infatti a garantire ai clienti progetti di qualità, nei tempi previsti e con costi ridotti evitando trasferte e lavorando principalmente dalla loro sede. 
Alcuni aziende poi si rivolgono a \azienda per consulenze e \emph{partnership} nel ramo dello sviluppo di prodotti dedicati a \emph{smartphone} e \emph{tablet}, dove il team aziendale eccelle particolarmente.

\section{Struttura del documento}
Il documento è stato strutturato per descrivere in maniera esaustiva il percorso di stage nell'azienda \azienda{}. I capitoli presenti sono i seguenti:

\begin{description}
    \item[{\hyperref[cap:contesto-aziendale]{Il primo capitolo}}] descrive l'azienda \azienda{}, a partire dalla sua storia e dai prodotti che essa offre, fino alle tecnologie che vengono utilizzare giornalmente e le metodologie interne adottate nello sviluppo dei prodotti software. 
    
    \item[{\hyperref[cap:progetto-stage]{Il secondo capitolo}}] approfondisce l'argomento dello stage, descrivendolo in modo dettagliato. Vengono inoltre riportati gli strumenti e le tecnologie utilizzate, i problemi affrontati e un breve riassunto sul prodotto ottenuto.
    
    \item[{\hyperref[cap:analisi]{Il terzo capitolo}}] riguarda l'analisi dei requisiti del progetto di stage, riportando una tabella con il loro tracciamento.
    
    \item[{\hyperref[cap:progettazione]{Il quarto capitolo}}] descrive l'attività di progettazione svolta per ottenere il prodotto finale. Viene descritto l'utilizzo di api.ai nel progetto e la sua integrazione nel software aziendale.
    
    \item[{\hyperref[cap:verfica]{Il quinto capitolo}}] contiene 
    
    \item[{\hyperref[cap:conclusioni]{Il sesto capitolo}}] contiene le mie valutazione finali sull'attività di stage svolta, sulle conoscenze acquisite e un riassunto degli obiettivi e requisiti raggiunti e soddisfatti.

\end{description}

Riguardo la stesura del testo, relativamente al documento sono state adottate le seguenti convenzioni tipografiche:
\begin{itemize}
	\item gli acronimi, le abbreviazioni e i termini ambigui o di uso non comune menzionati vengono definiti nel glossario, situato alla fine del presente documento;
	\item per la prima occorrenza dei termini riportati 
	nel glossario viene utilizzata la seguente nomenclatura: \emph{parola}\glsfirstoccur{};
	\item i termini in lingua straniera o facenti parti del gergo tecnico sono evidenziati con il carattere \emph{corsivo}.
\end{itemize}