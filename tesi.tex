        %%---------------------%%
        %%					   %%	
        %%	  TESI TRIENNALE   %%
        %%	   MAURO CARLIN    %%
        %%                     %%
        %%---------------------%%
\documentclass[10pt,                    % corpo del font principale
               a4paper,                 % carta A4
               twoside,                 % impagina per fronte-retro
               openright,               % inizio capitoli a destra
               english,                 
               italian,                 
               ]{book}    

\usepackage[utf8]{inputenc}             % codifica di input; anche [latin1] va bene
                                        % NOTA BENE! va accordata con le preferenze dell'editor

%**************************************************************
% Importazione package
%************************************************************** 

%\usepackage{amsmath,amssymb,amsthm}    % matematica

\usepackage[english, italian]{babel}    % per scrivere in italiano e in inglese;
                                        % l'ultima lingua (l'italiano) risulta predefinita

\usepackage{bookmark}                   % segnalibri

\usepackage{caption}                    % didascalie

\usepackage{chngpage,calc}              % centra il frontespizio

\usepackage{csquotes}                   % gestisce automaticamente i caratteri (")

\usepackage{emptypage}                  % pagine vuote senza testatina e piede di pagina

\usepackage{epigraph}					% per epigrafi

\usepackage{eurosym}                    % simbolo dell'euro

\usepackage[T1]{fontenc}                % codifica dei font:
                                        % NOTA BENE! richiede una distribuzione *completa* di LaTeX

%\usepackage{indentfirst}               % rientra il primo paragrafo di ogni sezione

\usepackage{graphicx}                   % immagini

\usepackage{hyperref}                   % collegamenti ipertestuali



\usepackage[binding=5mm]{layaureo}      % margini ottimizzati per l'A4; rilegatura di 5 mm

\usepackage{listings}                   % codici

\usepackage{microtype}                  % microtipografia

\usepackage{mparhack,fixltx2e,relsize}  % finezze tipografiche

\usepackage{nameref}                    % visualizza nome dei riferimenti                                      

\usepackage[font=small]{quoting}        % citazioni

\usepackage{subfigure}                  % sottofigure, sottotabelle

\usepackage[italian]{varioref}          % riferimenti completi della pagina

\usepackage[dvipsnames]{xcolor}         % colori

\usepackage{tabulary}
\usepackage{booktabs}                   % tabelle                                       
\usepackage{tabularx}                   % tabelle di larghezza prefissata                                    
\usepackage{longtable}                  % tabelle su più pagine                                        
\usepackage{ltxtable}                   % tabelle su più pagine e adattabili in larghezza
\newcolumntype{U}{>{\centering\arraybackslash}m{2cm}}
\newcolumntype{P}{>{\centering\arraybackslash}m{3cm}}
\newcolumntype{Z}{>{\centering\arraybackslash}m{4.2cm}}

\usepackage[toc, acronym]{glossaries}   % glossario
                                        % per includerlo nel documento bisogna:
                                        % 1. compilare una prima volta tesi.tex;
                                        % 2. eseguire: makeindex -s tesi.ist -t tesi.glg -o tesi.gls tesi.glo
                                        % 3. eseguire: makeindex -s tesi.ist -t tesi.alg -o tesi.acr tesi.acn
                                        % 4. compilare due volte tesi.tex.

\usepackage[backend=bibtex,style=verbose-ibid,hyperref,backref]{biblatex}
                                        % eccellente pacchetto per la bibliografia; 
                                        % produce uno stile di citazione autore-anno; 
                                        % lo stile "numeric-comp" produce riferimenti numerici
                                        % per includerlo nel documento bisogna:
                                        % 1. compilare una prima volta tesi.tex;
                                        % 2. eseguire: biber tesi
                                        % 3. compilare ancora tesi.tex.
                                        

\input{riferimenti}

\begin{document}
%**************************************************************
% Materiale iniziale
%**************************************************************
\frontmatter
\input{Contorno/frontespizio}
\input{Contorno/colophon}
% !TEX encoding = UTF-8
% !TEX TS-program = pdflatex
% !TEX root = ../tesi.tex

%**************************************************************
% Dedica
%**************************************************************
\cleardoublepage
\phantomsection
\thispagestyle{empty}
\pdfbookmark{Citazione}{Citazione}

\vspace*{3cm}

\begin{center}
I'm personally convinced that computer science has a lot in common with physics. Both are about how the world works at a rather fundamental level. The difference, of course, is that while in physics you're supposed to figure out how the world is made up, in computer science you create the world. Within the confines of the computer, you're the creator. You get to ultimately control everything that happens. If you're good enough, you can be God. On a small scale. \\ \medskip
--- Linus Torvalds
\end{center}

% !TEX encoding = UTF-8
% !TEX TS-program = pdflatex
% !TEX root = ../tesi.tex

%**************************************************************
% Sommario
%**************************************************************
\cleardoublepage
\phantomsection
\pdfbookmark{Sommario}{Sommario}
\begingroup
\let\clearpage\relax
\let\cleardoublepage\relax
\let\cleardoublepage\relax

\chapter*{Sommario}

Il presente documento descrive il lavoro svolto durante il periodo di stage dal laureando Mauro Carlin presso l'azienda \azienda{} di Belluno (BL). Lo stage è stato svolto alla conclusione del percorso di studi della Laurea Triennale ed è durato in totale 300 ore. Vengono riportate tutte le principali attività intraprese durante questo periodo, le difficoltà incontrate e una panoramica sul prodotto finale ottenuto\\
Gli obiettivi da raggiungere erano molteplici. In primo luogo l'azienda ha richiesto un'analisi dei principali strumenti per il \gls{NLP} presenti sul mercato, in modo da valutarne pregi e difetti. Questo strumento viene utilizzato per trasformare le domande di un utente in dati processabili.


Il passo successivo è stato studiare ed integrare questo sistema in due \glspl{chatbot} di Facebook Messenger creati e gestiti dall'azienda stessa, per dare la possibilità all'utente di interagire con essi anche tramite domande di senso compiuto, e non solo attraverso le possibilità offerte dalla piattaforma di Facebook.\\

%\vfill
%
%\selectlanguage{english}
%\pdfbookmark{Abstract}{Abstract}
%\chapter*{Abstract}
%
%\selectlanguage{italian}

\endgroup			

\vfill
% !TEX encoding = UTF-8
% !TEX TS-program = pdflatex
% !TEX root = ../tesi.tex

%**************************************************************
% Ringraziamenti
%**************************************************************
\cleardoublepage
\phantomsection
\pdfbookmark{Ringraziamenti}{ringraziamenti}
\begingroup
\let\clearpage\relax
\let\cleardoublepage\relax
\let\cleardoublepage\relax

\chapter*{Ringraziamenti}

\noindent \textit{Innanzitutto, vorrei esprimere la mia gratitudine al Prof. Paolo Baldan, relatore della mia tesi, per l'aiuto e il sostegno fornitomi durante la stesura del lavoro.}\\

\noindent \textit{Desidero ringraziare con affetto i miei genitori Rita e Roberto per il sostegno, il grande aiuto e per essermi stati vicini in ogni momento durante gli anni di studio.}\\

\noindent \textit{Ringrazio tutti gli amici conosciuti durante questo percorso, in particolare Luca, Pier Paolo, Nicola, Mattia, Marco, Simeone e Tommaso, per aver reso questi tre anni indimenticabili.}\\

\noindent \textit{Infine ringrazio tutti coloro che mi hanno aiutato, anche con un piccolo gesto, a raggiungere questo meraviglioso traguardo.}\\
\bigskip

\noindent\textit{\myLocation, \myTime}
\hfill \myName

\endgroup


\input{Contorno/indici}
\cleardoublepage

%**************************************************************
% Materiale principale
%**************************************************************
\mainmatter
% !TEX encoding = UTF-8
% !TEX TS-program = pdflatex
% !TEX root = ../tesi.tex

%**************************************************************
\chapter{Il contesto aziendale}
\label{cap:contesto-aziendale}
%**************************************************************

\section{Profilo aziendale}
\azienda{} è un'azienda che nasce nel 2003 a Belluno, dove tuttora mantiene la propria sede. Nei primi anni si dedica allo sviluppo di applicativo web per conto di terzi, lavorando in stretta collaborazione con aziende di comunicazione per consulenze di tipo tecnico. Parallelamente inizia la creazione del prodotto di punta dell'azienda: \textbf{SMSHosting}\footcite{smshosting}. Si tratta di un \emph{gateway} per l'invio e la ricezione di sms professionali da web, che consente agli utenti di comunicare con i propri clienti direttamente dalla propria area riservata, oppure dall'esterno tramite email , \gls{FTP}, moduli web, software Windows, app per \emph{smartphone} e molto altro.  \\
Nel corso degli anni, e con l'introduzione di nuove tecnologie, l'azienda evolve i propri prodotti, introducendo nuovi strumenti di comunicazione per i propri clienti, focalizzando le proprie competenze sui mobile che dal 2009 iniziavano a diffondersi.
Viene così creata una piattaforma dedicata all'invio di \textbf{messaggi \gls{push}} tramite le più diffuse \emph{messaging apps}, come Facebook Messenger e Telegram. I clienti possono utilizzare questo servizio, integrato nella piattaforma SMSHosting, attraverso email o delle semplici \gls{API} REST.\\
Come ultima novità \azienda{} ha iniziato lo sviluppo di \gls{chatbot} per Facebook Messenger e Telegram. L'azienda mette a disposizione sia una piattaforma semplice e intuitiva dove un utente può creare il proprio \gls{chatbot} in pochi passi, seguendo i più comuni template di business, sia la possibilità di crearne di nuovi secondo le richieste del cliente.

\section{Tecnologie utilizzate}
Le tecnologie principali che vengono utilizzate da \azienda{} per lo sviluppo dei propri prodotti possono essere divise in tre diverse aree:
\begin{itemize}
	\item applicazioni \emph{iOS} e \emph{Android} native;
	\item applicazioni web con tecnologie \emph{Java};
	\item \emph{frontend}.
\end{itemize} 

\subsection{Applicazioni \emph{iOS} e \emph{Android} native}
Le tecnologie utilizzate per lo sviluppo di applicazioni mobile dipende naturalmente dal sistema operativo dove si vuole sviluppare. L'azienda utilizza anche dei \emph{framework cross-platform} come \emph{PhoneGap}.
\subsubsection{Android}
Per quanto riguarda \emph{Android} l'azienda si affida al linguaggio nativo di questo sistema operativo, cioè \emph{Java}. 
Il team di sviluppo ha grande conoscenza di questo linguaggio, agevolando così lo sviluppo dei nuovi prodotti.
\subsubsection{iOS}
Per lo sviluppo di applicazioni \emph{iOS} il linguaggio che viene principalmente utilizzato è \emph{Objective-C}, un linguaggio ben noto a coloro che devono codificare questa versione dei prodotti.

\subsection{Applicazioni web con Java}
Le applicazioni web su piattaforma \emph{Java} sono nel DNA di \azienda{} fin dalla nascita nel 2003. \\
Il team di sviluppo ha grande conoscenza dei principali \emph{framework} di sviluppo ed ha lavorato a progetti di grande dimensione utilizzando sia CMS open source (Open CMS) che commerciali (Broadvision, Vignette OpenText).
Il \emph{framework} \emph{Spring}, \emph{Hibernate ORM}, \emph{Quartz Scheduler},\emph{ Jersey for RESTful Web services} sono solo pochi esempi delle librerie comunemente utilizzate e che fanno parte del \emph{core} dei loro prodotti.

\subsection{Frontend}
Nello sviluppo della parte \emph{frontend} dei propri portali e applicativi web, \azienda{} mira ad utilizzare gli strumenti più innovativi per garantire la massima velocità di presentazione e la compatibilità con i \emph{device} di nuova generazione. \\
I principali linguaggi e \emph{framework} sono \emph{HTML5, CSS3, JQuery, Bootstrap e Sencha}.

\section{Processi aziendali}
\subsection{Metodologia}
Fino dai suoi inizi \azienda{} crea delle soluzioni software altamente dipendenti dalle specifiche dei propri clienti. Per potere fare ciò è indispensabile mantenere una stretta comunicazione con il cliente, per capire le sue volontà e le sue richieste in modo preciso. La realizzazioni di nuovi progetti deve quindi essere in grado di reagire al cambiamento dei requisiti anche in fase di sviluppo, in quanto il cliente può cambiare idee e giudizi sulle funzionalità de proprio prodotto.\\
La metodologia di sviluppo più adatta, adottata quasi completamente dall'azienda, è la \gls{ASD}.
Il metodo agile utilizzato da \azienda{} prevede una forte e frequente collaborazione con il cliente, in modo da ricevere dei \emph{feedback} puntuali sugli incrementi portati al prodotto. In ogni momento il cliente si trova ad avere un prodotto via via più completo, che rispecchia i requisiti da esso imposti e discussi con l'azienda.\\
La chiave per il successo di questa metodologia è racchiusa in questi punti:
\begin{itemize}
	\item sviluppare qualcosa di utile;
	\item coltivare la fiducia degli \emph{stakeholders};
	\item costituire gruppi di lavoro competenti e collaborativi;
	\item far sì che il team abbia la possibilità e sia in grado di prendere decisioni;
	\item consegnare spesso nuove versioni all'aggiunta di nuove funzionalità;
	\item incoraggiare l'adattabilità;
	\item cercare di ottenere l'eccellenza tecnica;
	\item quando possibile, aumentare il volume di dati immessi.
\end{itemize}

\subsection{Strumenti a supporto dei processi}
\label{Teconologie}
\subsubsection{Gestione di progetto}
Per la \textbf{gestione di progetto} \azienda{} utilizza Asana\footcite{asana} per l'assegnazione di \emph{task} relativi a nuove attività da svolgere o \emph{bug} da correggere e BeeBole\footcite{beebole} come \emph{web timesheet}.
\paragraph{Asana}
Asana è un \gls{SaaS} \emph{web based} che mira a migliorare la collaborazione all'interno dei team di lavoro. Permette infatti la gestione di progetti e \emph{tasks} online, senza dover utilizzare le email.\\
\azienda{} ha creato un proprio \emph{workspace}, dove poter aggiungere nuovi progetti e nuovi \emph{tasks} assegnati a questi progetti. \\ \\
Per ogni \emph{task} che si vuole creare, è possibile specificare una serie di dettagli:
\begin{itemize}
	\item \textbf{nome e descrizione} relativa;
	\item il \textbf{membro del team} che ha il compito di svolgere quel task. La persona deve essere registrata all'interno del \emph{workspace} di \azienda;
	\item la \textbf{data} entro la quale il compito deve essere svolto. Una settimana prima l'incaricato riceverà una mail di promemoria;
	\item uno o più \textbf{\emph{tags}} per differenziare i diversi \emph{tasks} all'interno del progetto.
\end{itemize}
Una volta assegnato il \emph{task}, incaricato e assegnatario, più tutte le persone che sono in grado di visualizzarlo, potranno aprire una conversazione dedicata nella sezione specifica di quel \emph{task}, dove scambiarsi idee, file e molto altro. Terminato il lavoro, l'incaricato dovrà marcare il \emph{task} come concluso attraverso l'apposita spunta.

\paragraph{BeeBole}
BeeBole è uno strumento di \textbf{\emph{web timesheet}} che dà la possibilità ai propri utenti di monitorare in modo efficiente il tempo dedicato a progetti, clienti e incarichi. \\
L'azienda \azienda{} sfrutta questo applicativo per controllare il budget dedicato ad ogni progetto e le ore effettivamente investite da ogni componente del team, per procedere con una fatturazione corretta e trasparente nei confronti del cliente.
\subsubsection{Gestione di versione}
L'azienda \azienda{} utilizza come \textbf{sistema di controllo di versione} del codice \gls{Mercurial}. In particolare, per
poter raccogliere tutto il codice derivante dai vari progetti delle varie aree sviluppo, viene utilizzata la versione premium di BitBucket\footcite{bit}.\\

Alcuni vantaggi dell'adozione di un sistema di versionamento del codice sono:
\begin{itemize}
	\item \textbf{ridondanza}: ogni sviluppatore possiede un \emph{backup} della \emph{repository} localmente, limitando così la possibilità di perdita totale dei dati;
	\item \textbf{disponibilità}: anche in assenza di connessione alla \emph{repository} principale, è possibile continuare ad effettuare \emph{commit} ed a lavorare. Una volta ripristinata
la connessione, la \emph{repository} locale può essere sincronizzata con quella remota
rendendo le modifiche visibili a tutti;
	\item \textbf{\emph{branch e merge}}: permette con molta facilità la creazione dei cosiddetti \emph{branch}, ossia delle ramificazioni dal prodotto stabile presente nella \emph{repository}, per apportare modifiche o nuove funzionalità evitando di introdurre errori e bug. Una volta che anche questa nuova \emph{feature} risulta corretta, può essere inserita all'interno del progetto principale tramite un \emph{merge}.
\end{itemize}
\subsubsection{Comunicazione aziendale}
Come strumento di \textbf{comunicazione} tra i dipendenti dell'azienda, \azienda{} ha scelto Slack\footcite{slack}.
Slack è un software che rientra nella categoria degli strumenti di collaborazione aziendale utilizzato per inviare messaggi in modo istantaneo ai membri del team. Il suo punto di forza è la possibilità di creare dei \textbf{canali} dedicati a un progetto o ad un particolare argomento, dove vengono inseriti tutti o solo parte dei dipendenti aziendali. È possibile inoltre comunicare con il team anche attraverso chat individuali o chat con due o più membri.\\
La scelta è ricaduta su questo strumento anche per la sua facilità di utilizzo e la caratteristica di essere fruibile da tutti i dispositivi \emph{iOs, Android, Windows} come applicazione e da \emph{web browser}.

\section{Clienti}
Grazie ai prodotti offerti da \azienda{} la sua clientela spazia in molti ambiti diversi come automobilismo, banche, telecomunicazioni e molto altro. Queste aziende si rivolgono a \azienda{} sia per migliorare le comunicazione tra loro e la propria clientela, attraverso la piattaforma di SMSHosting e le sue numerose soluzioni, sia per proporre lo sviluppo di nuovi prodotti \emph{web based}. \\
Il team di \azienda{} ha lavorato, direttamente o tramite i loro partner, con clienti in tutta Italia. 
Grazie ad una consolidata modalità di lavoro riescono infatti a garantire ai clienti progetti di qualità, nei tempi previsti e con costi ridotti evitando trasferte e lavorando principalmente dalla loro sede. 
Alcuni aziende poi si rivolgono a \azienda per consulenze e \emph{partnership} nel ramo dello sviluppo di prodotti dedicati a \emph{smartphone} e \emph{tablet}, dove il team aziendale eccelle particolarmente.

\section{Struttura del documento}
Il documento è stato strutturato per descrivere in maniera esaustiva il percorso di stage nell'azienda \azienda{}. I capitoli presenti sono i seguenti:

\begin{description}
    \item[{\hyperref[cap:contesto-aziendale]{Il primo capitolo}}] descrive l'azienda \azienda{}, a partire dalla sua storia e dai prodotti che essa offre, fino alle tecnologie che vengono utilizzare giornalmente e le metodologie interne adottate nello sviluppo dei prodotti software. 
    
    \item[{\hyperref[cap:progetto-stage]{Il secondo capitolo}}] approfondisce il progetto di stage nella sua interezza. Vengono infatti descritti gli strumenti utilizzati dallo stagista, i problemi affrontati, la pianificazione del lavoro e gli obiettivi da raggiungere.
    
    \item[{\hyperref[cap:analisi]{Il terzo capitolo}}] riporta i requisiti funzionali 
    
    \item[{\hyperref[cap:progettazione]{Il quarto capitolo}}] contiene 
    
    \item[{\hyperref[cap:verfica]{Il quinto capitolo}}] contiene 
    
    \item[{\hyperref[cap:conclusioni]{Il sesto capitolo}}] contiene 

\end{description}

Riguardo la stesura del testo, relativamente al documento sono state adottate le seguenti convenzioni tipografiche:
\begin{itemize}
	\item gli acronimi, le abbreviazioni e i termini ambigui o di uso non comune menzionati vengono definiti nel glossario, situato alla fine del presente documento;
	\item per la prima occorrenza dei termini riportati 
	nel glossario viene utilizzata la seguente nomenclatura: \emph{parola}\glsfirstoccur{};
	\item i termini in lingua straniera o facenti parti del gergo tecnico sono evidenziati con il carattere \emph{corsivo}.
\end{itemize}             % Contesto aziendale
% !TEX encoding = UTF-8
% !TEX TS-program = pdflatex
% !TEX root = ../tesi.tex

%**************************************************************
\chapter{Progetto di stage}
\label{cap:progetto-stage}

\section{Descrizione del progetto}

\section{Principali problematiche}

\section{Strumenti utilizzati}

\section{Prodotto ottenuto}


             % Descrizione progetto
% !TEX encoding = UTF-8
% !TEX TS-program = pdflatex
% !TEX root = ../tesi.tex

%**************************************************************
\chapter{Analisi dei requisiti}
\label{cap:analisi}

\section{Analisi di mercato}
Il primo passo da compiere per iniziare lo sviluppo de prodotto è stato scegliere la piattaforma di \gls{NLP} migliore in base ai requisiti imposti dall'azienda. Le richieste fatte da \azienda{} riguardanti questo strumento erano le seguenti:
\begin{itemize}
	\item \textbf{costo}: il prezzo per il suo utilizzo doveva essere uguale a 0;
	\item \textbf{lingua}: deve supportare sia la lingua italiana, visto che al momento attuale i \gls{chatbot} sono implementati solo con quella;
	\item \textbf{documentazione}: il servizio deve essere ben documentato per permettere all'azienda, una volta finito il periodo di stage, di imparare ad utilizzarlo velocemente.
\end{itemize}

Questa attività di analisi di mercato si è rivelata quindi fondamentale per la buona riuscita del progetto, visto l'importanza che questo strumento avrebbe avuto nell'intero periodo di sviluppo. Le piattaforme da me studiate e analizzate sono riportate di seguito.

\subsubsection{IBM Watson Conversation}
IBM Watson Conversation\footcite{watson} è un prodotto della piattaforma IBM Watson, che attraverso IBM Cloud dà la possibilità di integrare i più potenti mezzi di AI nelle tue applicazioni. Il servizio di Conversation, oltre alla possibilità di creare \gls{chatbot} e agenti virtuali, può essere istruito ed interrogato per analizzare il testo posto in input, attraverso le \gls{API} messe a disposizione.\\
È possibile infatti creare dei workspace dedicati dove, attraverso \emph{intent} ed \emph{entities} creati e gestiti dallo sviluppatore, analizzare le domande poste dagli utenti, estraendo i dati che più interessano. L'integrazione con l'applicativo aziendale risultava semplice, grazie al SDK di Java\footcite{watsonSDK} messo a disposizione da IBM.

\begin{figure}[h]
	\centering
	\includegraphics[scale=0.15]{../Immagini/conversation_arch_overview.png}
	\caption{Funzionamento IBM Watson Converation}
\end{figure}

Per quanto riguarda le richieste dell'azienda:
\begin{itemize}
	\item la \textbf{lingua italiana} è supportata, e non in versione beta;
	\item la \textbf{documentazione} è chiara ed esaustiva, con dei video di esempio molto utili;
	\item esiste un piano di \textbf{costi} gratis, chiamato \emph{Lite}, che però dà la possibilità di creare un numero limitato di \emph{workspace}, \emph{intent} ed \emph{entity}, risultando troppo vincolante per i futuri utilizzi aziendali. Le soluzioni a pagamento non sono state prese in considerazione in quanto non percorribili per l'azienda, almeno in un primo momento di utilizzo di questi servizi.
\end{itemize}

\begin{figure}[h]
	\centering
	\includegraphics[scale=0.1]{../Immagini/watson_conversations_icon.png}
	\caption{Logo di IBM Watson Conversation}
\end{figure}

\subsubsection{wit.ai}
wit.ai\footcite{witai} è una società nata nell'ottobre del 2013 e acquisita da Facebook Inc. nel 2015.
L'obiettivo di wit.ai è quello di semplificare la creazione di applicazioni che prevedono interazioni testuali o vocali; per farlo viene messa a disposizione degli sviluppatori una piattaforma di linguaggio naturale aperta ed estensibile che ha la peculiarità di apprendere tramite ogni interazione eseguita.\\
wit.ai mette a disposizione un SDK gratuito ed \emph{open source} per il riconoscimento del linguaggio
naturale. Questa piattaforma è caratterizzata dall'utilizzo di \emph{context}, \emph{intent} ed \emph{entity} che sono
dei costrutti messi a disposizione per tradurre le richieste vocali dell'utente in dati processabili. In particolare il \emph{context} si utilizza per monitorare lo stato della conversazione tra l'utente e wit.ai.

\begin{figure}[h]
	\centering
	\includegraphics[scale=0.4]{../Immagini/witai_example.png}
	\caption{Esempio di creazione di un intento in wit.ai}
\end{figure}

Per quanto riguarda le richieste dell'azienda:
\begin{itemize}
	\item la \textbf{lingua italiana} è supportata;
	\item la \textbf{documentazione} è chiara ed esaustiva;
	\item l'utilizzo di wit.ai è completamente gratuito per progetti sia pubblici che privati.
\end{itemize}

Dal punto di vista tecnico l'unica mancanza di questo strumento, che ha influito nella decisione di non adottarlo, è la impossibilità di impostare delle \emph{required entity} all'interno degli \emph{intent}. Questo aspetto obbliga lo sviluppatore a introdurre dei controlli a livello di \emph{business logic}, che altrimenti non sarebbero necessari, come nel caso di altre piattaforme che saranno esposte successivamente.

\begin{figure}[h]
	\centering
	\includegraphics[scale=0.5]{../Immagini/witai.png}
	\caption{Logo di wit.ai}
\end{figure}

\subsubsection{Microsoft LUIS}
Microsoft LUIS (Language Understanding Intelligent Service)\footcite{luis} è un prodotto di \emph{Microsoft Azure}, dedicato a comprendere le richieste di una persona tramite un \emph{language model} (entity/intent). \\
Come nelle altre piattaforme lo sviluppatore può creare degli \emph{intents}, cioè delle categorie di azioni che l'utente può intraprendere, dove nelle frasi ad esse correlate vengono evidenziate le \emph{entities}, ossia i pezzi di informazione di interesse, per poi gestirle nella logica del \gls{chatbot}. LUIS inoltre mette a disposizione la possibilità di "marcare" le \emph{entity} come \emph{required}, al contrario di wit.ai, e anche la creazione di cosiddette \emph{composite entities}, che possono essere intese come il raggruppamento di più \emph{entity} in una unica.
\begin{figure}[h]
	\centering
	\includegraphics[scale=0.5]{../Immagini/lex_example.png}
	\caption{Utilizzo di Amazon Lex}
\end{figure}
Per quanto riguarda le richieste dell'azienda:
\begin{itemize}
	\item la \textbf{lingua italiana} è supportata;
	\item la \textbf{documentazione} è abbastanza chiara;
	\item esiste un piano \textbf{gratuito} di utilizzo di LUIS, con una limitazione del numero di chiamate alle API.
\end{itemize}

\begin{figure}[h]
	\centering
	\includegraphics[scale=0.25]{../Immagini/luis.jpg}
	\caption{Logo di Microsoft LUIS}
\end{figure}

\subsubsection{Amazon Lex}
Amazon Lex\footcite{lex} è un servizio per la creazione di interfacce di comunicazione tramite voce e testo per qualsiasi tipo di applicazione. Amazon Lex offre funzionalità avanzate di apprendimento approfondito per il riconoscimento vocale e la dettatura, nonché per il riconoscimento del linguaggio naturale e la comprensione di testi, consentendo la creazione di applicazioni coinvolgenti e conversazioni realistiche. Con Amazon Lex, le stesse tecnologie di apprendimento approfondito su cui si basa \emph{Amazon Alexa} sono disponibili a tutti gli sviluppatori, consentendo così la creazione di \gls{chatbot} sofisticati e naturali in modo semplice e veloce.\\
L'interfaccia grafica consente ci creare i propri intents in modo intuitivo, seguendo le linee generali delle altre piattaforme.

Per quanto riguarda le richieste dell'azienda:
\begin{itemize}
	\item la \textbf{lingua italiana} è supportata;
	\item la \textbf{documentazione} non è chiara, sopratutto per quanto riguarda la creazione di \emph{intent}, \emph{entity} e \emph{utterance};
	\item esiste un piano \textbf{gratuito} per il primo anno di utilizzo di Amazon Lex. Finito questo tempo, la piattaforma diventa a pagamento, in proporzione al numero di chiamate effettuate.
\end{itemize}

\begin{figure}[h]
	\centering
	\includegraphics[scale=0.5]{../Immagini/amazon-lex.png}
	\caption{Logo di Amazon Lex}
\end{figure}

\subsubsection{Api.ai}
Api.ai\footcite{apiai} è uno degli strumenti con il maggior numero di \emph{features} per quanto riguarda il \emph{machine learning} e il \gls{NLP}. Una volta acquistato da Google nel 2016 il suo volume di utilizzatori è aumentato in maniera esponenziale. \\
Api.ai fornisce SDK per i principali linguaggi di programmazione tra i quali \emph{C++, C\#, Java, Node.js, JavaScript} e \emph{Python}. Inoltre può essere integrato con \emph{Amazon Echo} e \emph{Microsoft Cortana}. Le applicazioni sviluppate su questa piattaforma sono costituite da \emph{agent}, i quali si occupano di trasformare il linguaggio naturale in dati processabili. Tali \emph{agent} sono a loro volta costituiti da \emph{intent}, che hanno il compito di associare la richiesta dell'utente ad una determinata azione del software, ed \emph{entity}, che sono strumenti per estrarre dal linguaggio naturale i parametri attesi.

Per quanto riguarda le richieste dell'azienda:
\begin{itemize}
	\item la \textbf{lingua italiana} è supportata;
	\item la \textbf{documentazione} è molto chiara;
	\item il suo utilizzo è gratuito, con delle limitazioni per il numero di richieste al minuto, ora, giorno e mese.
\end{itemize}

\begin{figure}[h]
	\centering
	\includegraphics[scale=0.25]{../Immagini/apiai.png}
	\caption{Logo di api.ai}
\end{figure}

\subsection{Conclusioni dell'analisi}
Dopo una mia attenta analisi sui pregi e difetti di tutti gli strumenti, in accordo con il tutor aziendale, è stata fatta una breve riunione con gli altri dipendenti, dove ho illustrato in modo sintetico i dati raccolti. Amazon Lex e IBM Watson Conversation sono stati considerati non adeguati per lo sviluppo de progetto a causa dei loro costi, mentre le altre piattaforme, non presentando queste limitazioni, potevano essere tutte adottate. \\
La mia proposta è stata quella di utilizzare Api.ai, per la maggiore stabilità rispetto a wit.ai, per il maggior numero di lingue disponibili in vista di un supporto futuro ad interazioni con utenti di diversa nazionalità e per la maggiore maturità della piattaforma. L'azienda dopo aver valutato anch'essa i dati raccolti ha deciso di approvare la mia proposta, in quanto api.ai rispettava tutte le sue richieste iniziali, mostrando delle potenzialità molto interessanti per lo sviluppo del prodotto.             % Analisi requisiti
% !TEX encoding = UTF-8
% !TEX TS-program = pdflatex
% !TEX root = ../tesi.tex

%**************************************************************
\chapter{Progettazione}
\label{cap:progettazione}

La progettazione nel caso del prodotto che dovevo sviluppare si è concentrata particolarmente sulla piattaforma di api.ai, in quanto rappresentava la maggior parte del lavoro. Successivamente sono passato alla progettazione delle classi da introdurre nel codice già creato dall'azienda, per gestire le nuove funzionalità del \gls{chatbot}.

\section{Studio di api.ai}
Prima di passare all'attività di progettazione è stato fondamentale analizzare e studiare a fondo le possibilità che api.ai mette a disposizione per la creazione del prodotto a me richiesto. I concetti base per capire il funzionamento di api.ai sono quattro:
\begin{itemize}
	\item \textbf{agent}: gli \emph{agents} sono meglio descritti come moduli NLU (Natural Language Understanding). Questi possono essere inclusi nell'applicazione, nel prodotto o nel servizio e trasformano le richieste di utenti naturali in dati attivi. Questa trasformazione si verifica quando un input utente corrisponde a uno degli \emph{intent} all'interno dell'\emph{agent};
	\item \textbf{intent}: sono una mappatura tra quello che l'utente può scrivere in input e l'azione che il software deve intraprendere. Un intent è formato dalle seguenti sezioni:
	\begin{itemize}
		\item\textbf{ \emph{user says}}: perché l'\emph{agent} capisca la domanda, sono necessari esempi di come la stessa domanda può essere posta in modi diversi. Lo sviluppatore aggiunge queste permutazioni alla sezione \emph{user says} dell'\emph{intent}. Più variazioni vengono aggiunte all'\emph{intent}, meglio l'\emph{agent} comprenderà l'utente;
		\item \textbf{\emph{action}}: contiene il nome della \emph{action}, che può essere utilizzato per attivare una particolare funzione del prodotto, e la tabella dei \textbf{\emph{parameters}}. I parameters possono gli elementi che collegano le parole nelle \emph{user says} alle \emph{entities};
		\item \textbf{\emph{response}}: in questa sezione è possibile definire la risposta di api.ai quando l'\emph{intent} viene attivato. Non è stato quasi mai utilizzato, in quanto la risposta all'utente veniva generata nella \emph{business logic}.
	\end{itemize}
	\item \textbf{context}: i \emph{context} rappresentano il contesto corrente della richiesta di un utente. Ciò è utile per differenziare frasi che possono essere vaghe o avere significati diversi a seconda delle preferenze dell'utente, della posizione geografica, della pagina corrente di un'applicazione o dell'argomento della conversazione. È possibile impostare un \emph{lifespan} ad ognuno di essi per definire dopo quante richieste il \emph{context} deve scadere;
	\item \textbf{entity}: le entities sono strumenti potenti utilizzati per estrarre i valori dei parametri dagli input degli utenti. Tutti i dati importanti che si desidera ottenere dalla richiesta di un utente, avranno un'entità corrispondente. Le \emph{system entities} sono entità pre-costruite fornite da API.AI per facilitare la gestione dei più comuni concetti (luoghi, orari, colori, ecc..). È possibile poi definire le proprie \emph{entities} in base alle necessità dello sviluppatore;
\end{itemize}


\section{Progettazione agents api.ai}
Durante la progettazione degli \emph{agents} per api.ai è stato necessario definire tutti gli \emph{intents} utili a soddisfare i requisiti definiti durante l'analisi dei requisiti. Il passo successivo è stato quello di progettare le \emph{user says} per ogni \emph{intent} e le relative \emph{entity}.

\subsubsection{Gestore di eventi}
Per quanto riguarda la progettazione del \gls{chatbot} dedicato alla gestione di eventi, gli \emph{intents} che mi sono serviti per soddisfare tutti i requisiti sono stati i seguenti:
\begin{itemize}
	\item \textbf{durata\_conferenza}: permette all'utente di domandare la durata di una conferenza e viene attivato con domande come: \emph{"Quanto dura la conferenza Y?"}. La risposta del \gls{chatbot} contiene il nome, l'inizio, la fine e la durata (in minuti o in ore) della conferenza richiesta dall'ospite;
	\begin{figure}[h]
		\centering
		\includegraphics[scale=0.12]{../Immagini/meteo_scelta.png}
		\caption{dfsf}
	\end{figure}
	\item \textbf{luogo\_conferenza}: permette all'utente di domandare il luogo dove si svolgerà la conferenza e viene attivato con domande come: \emph{"Dove si svolge la conferenza Y?"}. La risposta del \gls{chatbot} contiene un carosello predefinito da Messenger, con tutte le informazioni sull'aula in questione;
	\item \textbf{ora\_conferenza}: permette all'utente di l'orario di inizio e di fine di una conferenza e viene attivato con domande come: \emph{"A che ore inizia la conferenza Y?"}. La risposta del \gls{chatbot} contiene un carosello predefinito da Messenger, con tutte le informazioni sulla conferenza;
	\item \textbf{indicazioni\_stanza}: permette all'utente di domandare le indicazioni per trovare una determinata aula e viene attivato con domande come: \emph{"Dammi delle indicazioni per la stanza X"}. La risposta del \gls{chatbot} contiene le indicazioni presenti nel database, con una piccola mappa illustrativa;
	\item \textbf{programma\_giornata}: permette all'utente di domandare il il programma dell'evento di un determinato giorno e viene attivato con domande come: \emph{"Qual è il programma di oggi?"}. La risposta del \gls{chatbot} contiene un carosello per ogni conferenza in programma quel giorno;
	\item \textbf{programma\_no\_data}:
	\item \textbf{data\_ora\_stanza\_conferenza}:
	\item \textbf{data\_stanza\_conferenza}:
	\item \textbf{ora\_stanza\_conferenza}:
	\item \textbf{visualizza\_agenda}:
	\item \textbf{richiesta\_aiuto}:
\end{itemize}

\subsection{Dati ARPA Veneto}
Il \gls{chatbot} dedicato al meteo è stato realizzato grazie agli \emph{open data} messi a disposizione dall'Agenzia Regionale per la Prevenzione e Protezione Ambientale del Veneto(ARPAV). Ogni giorno, nel sito ufficiale\footcite{arpav}, vengono emessi tre bollettini:
\begin{itemize}
	\item \textbf{alle 9:00}: che rappresenta un aggiornamento del bollettino del giorno precedente;
	\item \textbf{alle 13:00}: il nuovo bollettino;
	\item \textbf{alle 16:00}: un aggiornamento del bollettino emesso alle 13.
\end{itemize} 

Il file XML che è possibile scaricare, contiene queste informazioni:
\begin{itemize}
	\item le previsioni dei cinque giorni successivi per le 18 zone in cui è stata divisa la regione del Veneto;
	\item una descrizione dell'evoluzione generale dei cinque giorni successivi, per tre macro zone: la regione intera, la zona delle Dolomiti e la pianura veneta.
\end{itemize}

Ad ogni nuova emissione del bollettino, i nuovi dati vengono inseriti nel database aziendale, in modo da comunicare agli utenti solamente le notizie più aggiornate.

\subsubsection{Meteo Veneto Bot}

Gli intents che ho deciso di creare per soddisfare tutti i requisiti sono i seguenti:
\begin{itemize}
	\item \textbf{richiesta\_meteo}: permette all'utente di chiedere le previsioni del meteo specificando una giornata o un periodo di tempo (es. weekend) e il comune di interesse (se non viene specificato, si considera il comune da lui selezionato all'inizio dell'interazione con il \gls{chatbot}). La risposta contiene un carosello con il meteo richiesto.
	\begin{figure}[h!]
		\centering
		\includegraphics[scale=0.12]{../Immagini/richiesta_meteo.png}
		\caption{Esempio di }
	\end{figure}	
	\item \textbf{richiesta\_sole}: permette all'utente di chiedere se è previsto il sole in una specifica giornata o un periodo di tempo (es. weekend), in un determinato comune. La risposta è formata da due messaggi: il primo mostra le giornate dove è previsto il sole, tra quelle richieste dall'utente, il secondo contiene i caroselli delle previsioni.
	\begin{figure}[h!]
		\centering
		\includegraphics[scale=0.12]{../Immagini/richiesta_sole.png}
		\caption{Esempio di }
	\end{figure}	
	\item \textbf{richiesta\_pioggia}: permette all'utente di chiedere se è prevista pioggia in una specifica giornata o un periodo di tempo (es. weekend), in un determinato comune. La risposta è formata da due messaggi: il primo mostra le giornate dove è prevista pioggia, tra quelle richieste dall'utente, il secondo contiene i caroselli delle previsioni.
	\begin{figure}[h!]
		\centering
		\includegraphics[scale=0.12]{../Immagini/richiesta_pioggia.png}% "%" necessario
		\qquad\qquad
		\includegraphics[scale=0.12]{../Immagini/richiesta_pioggia2.png}
		\caption{Didascalia comune alle due figure}
	\end{figure}
	\item \textbf{richiesta\_nebbia}: permette all'utente di chiedere se è prevista nebbia in una specifica giornata o un periodo di tempo (es. weekend), in un determinato comune. La risposta è formata da due messaggi: il primo mostra le giornate dove è prevista nebbia, tra quelle richieste dall'utente, il secondo contiene i caroselli delle previsioni.
	\begin{figure}[h!]
		\centering
		\includegraphics[scale=0.12]{../Immagini/richiesta_nebbia.png}
		\caption{Esempio di }
	\end{figure}
	\item \textbf{richiesta\_neve}: permette all'utente di chiedere se è prevista neve in una specifica giornata o un periodo di tempo (es. weekend), in un determinato comune. La risposta è formata da due messaggi: il primo mostra le giornate dove è prevista neve, tra quelle richieste dall'utente, il secondo contiene i caroselli delle previsioni.
	\begin{figure}[h!]
		\centering
		\includegraphics[scale=0.12]{../Immagini/richiesta_nebbia.png}
		\caption{Esempio di }
	\end{figure}
	\item \textbf{richiesta\_bel\_tempo}: permette all'utente di chiedere se è previsto bel tempo in una specifica giornata o un periodo di tempo (es. weekend), in un determinato comune. La risposta è formata da due messaggi: il primo mostra le giornate dove è previsto bel tempo, tra quelle richieste dall'utente, il secondo contiene i caroselli delle previsioni.
	\begin{figure}[h!]
		\centering
		\includegraphics[scale=0.12]{../Immagini/richiesta_bel_tempo.png}
		\caption{Esempio di }
	\end{figure}
	\item \textbf{richiesta\_brutto\_tempo}: permette all'utente di chiedere se è previsto brutto tempo in una specifica giornata o un periodo di tempo (es. weekend), in un determinato comune. La risposta è formata da due messaggi: il primo mostra le giornate dove è previsto brutto tempo, tra quelle richieste dall'utente, il secondo contiene i caroselli delle previsioni.
	\begin{figure}[h!]
		\centering
		\includegraphics[scale=0.12]{../Immagini/richiesta_brutto_tempo.png}
		\caption{Esempio di }
	\end{figure}
	\item \textbf{richiesta\_temperature}: permette all'utente di chiedere le temperature previste in una specifica giornata o un periodo di tempo (es. weekend), in un determinato comune. La risposta contiene le temperature massimi e minime previste fornite da ARPA Veneto.
	\begin{figure}[h!]
		\centering
		\includegraphics[scale=0.12]{../Immagini/richiesta_brutto_tempo.png}
		\caption{Esempio di }
	\end{figure}
	\item \textbf{ascolta\_bollettino}: permette all'utente di chiedere il bollettino audio emesso da ARPA Veneto ogni giorno. La risposta contiene il file audio richiesto.
	\begin{figure}[h!]
		\centering
		\includegraphics[scale=0.12]{../Immagini/bollettino.png}
		\caption{Esempio di }
	\end{figure}
	\item \textbf{fenomeni\_particolari}: permette all'utente di chiedere se sono presenti avvisi o fenomeni particolari emessi da ARPAV. La risposta contiene questi avvisi, se presenti.
	\begin{figure}[h!]
		\centering
		\includegraphics[scale=0.12]{../Immagini/bollettino.png}
		\caption{Esempio di }
	\end{figure}

\end{itemize}


\section{Jaro Winkler distance}
Durante l'attività di progettazione mi sono reso conto come fosse necessario gestire un possibile errore di scrittura dell'utente in una delle sue domande, soprattutto nelle parole fondamentali per formulare le risposte, come ad esempio il nome di un comune per il \gls{chatbot} del meteo o il nome di una conferenza in quello degli eventi. In un primo momento infatti l'input dell'utente veniva utilizzato direttamente nelle \emph{query SQL} per interrogare il database ed ottenere i dati di interesse, soprattutto attraverso l'operatore \emph{"LIKE"}. In questo modo però non è possibile gestire il caso in cui un utente scriva ad esempio il comune "Padvoa", intendendo Padova. \\
Per ovviare a questo problema quindi è stato deciso di introdurre, dopo uno studio delle possibili soluzioni, la Jaro Winkler distance\footcite{jaro}, ossia una metrica che misura la "distanza" tra due stringhe per capire quanto esse siano simili tra loro. Grazie a questa accortezza, nel caso di errore di scrittura, il \gls{chatbot} è in grado di:
\begin{itemize}
	\item fornire una serie di opzioni di cosa secondo lui l'utente voleva scrivere, dando la possibilità ad esso di selezionare quella giusta;
	\item fornire i dati richiesti dall'utente nel caso ci sia un'unica corrispondenza simile a quanto scritto dall'utente nel database.
\end{itemize}        	    % Progettazione
% !TEX encoding = UTF-8
% !TEX TS-program = pdflatex
% !TEX root = ../tesi.tex

%**************************************************************
\chapter{Verifica e validazione}
\label{cap:verifica}
La \textbf{verifica} e \textbf{validazione} di un prodotto software hanno lo scopo di accertare che esso rispecchi i requisiti e che li rispetti nella maniera dovuta. Con l’attività di \textbf{verifica} viene accertato che lo stato di avanzamento del prodotto soddisfi i requisiti precedentemente fissati. Grazie a questa attività è possibile accertare la corretta costruzione del software. L’attività di \textbf{validazione} ha invece lo scopo di accertare che il prodotto finale corrisponda alle attese in modo da soddisfare tutti i requisiti prefissati in fase di analisi.\\ \\
Durante il mio stage lo sviluppo delle nuove funzionalità ha portato alla minimizzazione dei tempi di verifica e validazione. Tale decisione è stata presa in comune accordo con il tutor aziendale, in quanto lo scopo dello stage mirava all’estensione di più funzionalità possibili, che potranno essere testate successivamente dall’azienda.
Questa decisione è inoltre appoggiata dalla metodologia \emph{agile} utilizzata nello sviluppo del progetto, la quale definisce la qualità del software come la capacità di soddisfare i bisogni del cliente piuttosto che soddisfare metriche fissata a priori.
In ogni caso, vista l'importanza di queste attività, sono state adottate tecniche di analisi statica e analisi dinamica per il codice sorgente, al fine di verificarne e validarne il comportamento.
\begin{itemize}
	\item \textbf{analisi statica}: l'analisi statica è il processo di valutazione di un sistema o di un suo componente basato sulla sua forma, struttura, contenuto, documentazione senza che esso sia eseguito. Gli strumenti di analisi statica del codice consentono di individuare porzioni di codice del proprio programma ad alta probabilità di errore. Avendo a disposizione una lista di linee di codice sospette, un programmatore può poi verificare se siano presenti errori e, in caso positivo, rivedere il codice corrispondente e correggere le problematiche individuate;
	\item \textbf{analisi dinamica}: l'analisi dinamica è il processo di valutazione di un sistema software o di un suo componente basato sulla osservazione del suo comportamento in esecuzione.
\end{itemize}


\section{Verifica e validazione dei chatbots}
Gran parte del progetto di stage verteva nell'istruire gli \emph{agent} di api.ai per gestire le domande degli utenti fatte attraverso i \gls{Chatbot}. Essendo quindi una parte molto importante del prodotto sono stati creati alcuni test per verificare e validare le sue funzionalità. Questi test non sono stati automatizzati in quanto si è deciso di svolgerli attraverso due pagine Facebook create dall'azienda dove sono stati associati i due \glspl{Chatbot} di prova.\\
Si è quindi pensato insieme al tutor di impostare una serie di domande da porre al bot, cercando di coprire tutti i requisiti che esso dovesse soddisfare. \\
Grazie a questi controlli è stato possibile accertare che entrambi i \glspl{Chatbot} soddisfano tutti i requisiti obbligatori e desiderabili esposti nella sezione \ref{requisiti}.

				% Testing e realizzione
% !TEX encoding = UTF-8
% !TEX TS-program = pdflatex
% !TEX root = ../tesi.tex

%**************************************************************
\chapter{Conclusioni}
\label{cap:conclusioni}

\section{Consutivo finale}


\section{Raggiungimento degli obiettivi}
Come descritto nella sezione \ref{obiettivi}, io e il mio tutor aziendale abbiamo stilato una serie di obiettivi da raggiungere nelle 300 ore di stage, divisi in obiettivi obbligatori e desiderabili. \\ 
Gli obiettivi obbligatori si concentravano sull'analisi di mercato dei principali strumenti per il \gls{NLP} e sullo sviluppo delle funzionalità del prodotto, integrandole nell'architettura già creata dall'azienda. Gli obiettivi desiderabili si focalizzavano invece nel testare in modo approfondito il risultato ottenuto, attraverso l'utilizzo dei \gls{chatbot} di Facebook Messenger in un ambiente locale. 
Al termine dello stage tutti gli obiettivi prefissati sono stati soddisfatti nei tempi previsti. Questo è stato possibile grazie alla buona pianificazione delle tempistiche necessarie allo svolgimento dei vari compiti, sia da parte dello stagista che da parte dell'azienda. La metodologia di sviluppo \emph{Agile} ha permesso si è inoltre dimostrata molto efficace, permettendomi di rispondere in modo rapido ai cambiamenti proposti dall'azienda durante il mio lavoro. 
\section{Conoscenze acquisite}
Da un punto di vista formativo l'attività di stage è stata sicuramente molto positiva. Queste 300 ore mi hanno permesso di mettermi alla prova, dandomi un riscontro su quanto gli anni universitari mi hanno preparato, sia nell'ambito delle mie conoscenze, sia da quello umano, per affrontare il mondo del lavoro.


Lo stage ha sicuramente arricchito il mio bagaglio personale di competenze tecniche, dandomi una panoramica delle procedure e delle attività che giornalmente si svolgono all'interno di un'azienda. Questo mi ha inoltre permesso di rendermi conto quanto una buona collaborazione tra i membri di un team sia fondamentale nello sviluppo di un prodotto, di qualsiasi tipo esso si tratti.
Le competenze acquisite non riguardano solamente il campo informatico, grazie alle nuove tecnologie utilizzate, ma anche le mie capacità organizzative, fondamentali per il rispetto dei vincoli e delle scadenze imposte dall'azienda. Ho dovuto infatti pianificare in modo preciso il lavoro delle mie settimane, per garantire ad \azienda{} la portata a termine del mio progetto, senza dover impiegare un dipendente per adempire alle mie mancanze. 

\section{Valutazione personale}
Nel complesso ritengo la mia esperienza di stage molto positiva ed istruttiva. L'inserimento, se pur breve, in un contesto aziendale permette di far capire ad uno studente quanto siano differenti il mondo del lavoro e quello universitario, in modo da cogliere gli aspetti fondamentali che solamente esperienze di questo tipo ti possono dare.


L'aspetto più importante di cui mi sono reso conto in questo periodo, è quanto il mio background universitario mi abbia aiutato nell'affrontare le sfide che ogni giorno si sono presentate. L'approccio e la metodologia che i vari corsi di studio ci hanno insegnato si sono rivelati preziosi ed indispensabili per il buon svolgimento di questo stage, dimostrando come il sapersi adattare a nuove situazioni, permetta di superare le lacune tecniche che, in un mondo come quello dell'informatica, si presentano spesso.     			% Conclusioni
%**************************************************************
% Materiale finale
%**************************************************************
\backmatter

\printglossaries
\input{Contorno/bibliografia}
\end{document}