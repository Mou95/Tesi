% !TEX encoding = UTF-8
% !TEX TS-program = pdflatex
% !TEX root = ../tesi.tex

%**************************************************************
% Sommario
%**************************************************************
\cleardoublepage
\phantomsection
\pdfbookmark{Sommario}{Sommario}
\begingroup
\let\clearpage\relax
\let\cleardoublepage\relax
\let\cleardoublepage\relax

\chapter*{Sommario}

Il presente documento descrive il lavoro svolto durante il periodo di stage dal laureando Mauro Carlin presso l'azienda i-contact S.r.l di Belluno (BL). Lo stage è stato svolto alla conclusione del percorso di studi della Laurea Triennale ed è durato in totale 300 ore.\\
Gli obiettivi da raggiungere erano molteplici. In primo luogo l'azienda ha richiesto un'analisi dei principali \gls{nlp} presenti sul mercato, in modo da valutarne pregi e difetti. Questo strumento viene utilizzato per trasformare le domande di un utente in dati processabili.\\
Il passo successivo è stato studiare ed integrare questo sistema in due chatbot di Facebook Messenger creati e gestiti dall'azienda stessa, per dare la possibilità all'utente di interagire con essi anche tramite domande di senso compiuto, e non solo tramite le possibilità offerte dalla piattaforma Facebook.\\
Il presente documento descrive nei primi due capitoli il contesto aziendale e come questo progetto di stage possa essere stato utile per l'azienda. Il terzo capitolo documenta lo svolgimento dello stage descrivendo le attività che sono state portate a termine, i punti salienti del progetto stesso e le principali scelte progettuali. Il quarto ed ultimo capitolo presenta infine una valutazione dello svolgimento dello stage rispetto agli obiettivi aziendali e alle conoscenze acquisite dallo studente. 

%\vfill
%
%\selectlanguage{english}
%\pdfbookmark{Abstract}{Abstract}
%\chapter*{Abstract}
%
%\selectlanguage{italian}

\endgroup			

\vfill