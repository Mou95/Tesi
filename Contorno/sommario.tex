% !TEX encoding = UTF-8
% !TEX TS-program = pdflatex
% !TEX root = ../tesi.tex

%**************************************************************
% Sommario
%**************************************************************
\cleardoublepage
\phantomsection
\pdfbookmark{Sommario}{Sommario}
\begingroup
\let\clearpage\relax
\let\cleardoublepage\relax
\let\cleardoublepage\relax

\chapter*{Sommario}

Il presente documento descrive il lavoro svolto durante il periodo di stage dal laureando Mauro Carlin presso l'azienda \azienda{} di Belluno (BL). Lo stage è stato svolto alla conclusione del percorso di studi della Laurea Triennale ed è durato in totale 300 ore. Vengono riportate tutte le principali attività intraprese durante questo periodo, le difficoltà incontrate e una panoramica sul prodotto finale ottenuto\\
Gli obiettivi da raggiungere erano molteplici. In primo luogo l'azienda ha richiesto un'analisi dei principali strumenti per il \gls{NLP} presenti sul mercato, in modo da valutarne pregi e difetti. Questo strumento viene utilizzato per trasformare le domande di un utente in dati processabili.


Il passo successivo è stato studiare ed integrare questo sistema in due \glspl{chatbot} di Facebook Messenger creati e gestiti dall'azienda stessa, per dare la possibilità all'utente di interagire con essi anche tramite domande di senso compiuto, e non solo attraverso le possibilità offerte dalla piattaforma di Facebook.\\

%\vfill
%
%\selectlanguage{english}
%\pdfbookmark{Abstract}{Abstract}
%\chapter*{Abstract}
%
%\selectlanguage{italian}

\endgroup			

\vfill