%**************************************************************
% file contenente le impostazioni della tesi
%**************************************************************


%**************************************************************
% Frontespizio
%**************************************************************

% Autore
\newcommand{\myName}{Mauro Carlin}                                    
\newcommand{\myTitle}{Chatbot di Facebook Messenger per la risposta a domande frequenti}

% Tipo di tesi                   
\newcommand{\myDegree}{Tesi di laurea triennale}

% Università             
\newcommand{\myUni}{Università degli Studi di Padova}

% Facoltà       
\newcommand{\myFaculty}{Corso di Laurea in Informatica}

% Dipartimento
\newcommand{\myDepartment}{Dipartimento di Matematica "Tullio Levi-Civita"}

% Titolo del relatore
\newcommand{\profTitle}{Prof. }

% Relatore
\newcommand{\myProf}{Paolo Baldan}

% Luogo
\newcommand{\myLocation}{Padova}

% Anno accademico
\newcommand{\myAA}{2016-2017}

% Data discussione
\newcommand{\myTime}{Settembre 2017}


%**************************************************************
% Impostazioni di impaginazione
% see: http://wwwcdf.pd.infn.it/AppuntiLinux/a2547.htm
%**************************************************************

\setlength{\parindent}{14pt}   % larghezza rientro della prima riga
\setlength{\parskip}{0pt}   % distanza tra i paragrafi


%**************************************************************
% Impostazioni di biblatex
%**************************************************************
\bibliography{bibliografia} % database di biblatex 

\defbibheading{bibliography} {
    \cleardoublepage
    \phantomsection 
    \addcontentsline{toc}{chapter}{\bibname}
    \chapter*{\bibname\markboth{\bibname}{\bibname}}
}

\setlength\bibitemsep{1.5\itemsep} % spazio tra entry

\DeclareBibliographyCategory{opere}
\DeclareBibliographyCategory{web}

\addtocategory{opere}{womak:lean-thinking}
\addtocategory{web}{site:agile-manifesto}

\defbibheading{opere}{\section*{Riferimenti bibliografici}}
\defbibheading{web}{\section*{Siti Web consultati}}


%**************************************************************
% Impostazioni di caption
%**************************************************************
\captionsetup{
    tableposition=top,
    figureposition=bottom,
    font=small,
    format=hang,
    labelfont=bf
}

%**************************************************************
% Impostazioni di glossaries
%**************************************************************

%**************************************************************
% Glossario
%**************************************************************
%\renewcommand{\glossaryname}{Glossario}

\newglossaryentry{ASD}
{
    name=\glslink{ASD}{Adaptive Software Development},
    text=Adaptive Software Development,
    description={\emph{L’Adaptive Software Development} (ASD) è una
metodologia composta da un insieme di regole di sviluppo software inserite in un
sistema complessivo detto \emph{Agile Project Management} i cui concetti base sono tre:
\begin{itemize}
	\item \emph{Leadership-Collaboration Management} - Uno stile di gestione misto fra gerarchico e collaborativo;
	\item \emph{emphFrom Processes to Pattern} - Passaggio dall’idea di processo definito e
misurabile a quella di processo non perfettamente definito, quasi un processo \emph{fuzzy};
	\item \emph{Peering into the Future} - Osservazione del futuro per capire come l’idea che produrrà un affare di successo debba essere legata al momento in cui diventerà una forma di business.
\end{itemize}}
}

\newglossaryentry{API}
{
    name=\glslink{API}{Application Program Interface},
    text=API,
    description={in informatica con il termine \emph{Application Programming
Interface} API (ing. interfaccia di programmazione di un’applicazione) si
indica ogni insieme di procedure disponibili al programmatore, di solito raggruppate
a formare un set di strumenti specifici per l’espletamento di un determinato
compito all’interno di un certo programma. La finalità è ottenere un’astrazione,
di solito tra l’hardware e il programmatore o tra software a basso e quello ad alto
livello semplificando così il lavoro di programmazione}
}

\newglossaryentry{AI}
{
    name=\glslink{AI}{Artificial Intelligence},
    text=AI,
    description={Definizioni specifiche possono essere date focalizzandosi o sui processi interni di ragionamento o sul comportamento esterno del sistema intelligente ed utilizzando come misura di efficacia o la somiglianza con il comportamento umano o con un comportamento ideale, detto razionale:
\begin{itemize}
	\item agire umanamente: il risultato dell’operazione compiuta dal sistema intelligente non è distinguibile da quella svolta da un umano.
	\item pensare umanamente: il processo che porta il sistema intelligente a risolvere un problema ricalca quello umano. Questo approccio è associato alle scienze cognitive.
	\item pensare razionalmente: il processo che porta il sistema intelligente a risolvere un problema è un procedimento formale che si rifà alla logica.
	\item agire razionalmente: il processo che porta il sistema intelligente a risolvere il problema è quello che gli permette di ottenere il miglior risultato atteso date le informazioni a disposizione
\end{itemize}}
}

\newglossaryentry{brainstorming}
{
    name=\glslink{brainstorming}{Brainstorming},
    text=brainstorming,
    description={L'espressione brainstorming, o brain storming (traducibile in lingua italiana come assalto mentale), è una tecnica creativa di gruppo per far emergere idee volte alla risoluzione di un problema}
}

\newglossaryentry{chatbot}
{
    name=\glslink{chatbot}{Chatbot},
    text=chatbot,
    description={in informatica con il termine \emph{Chatbot} si indica un programma che simula una conversazione tra un robot e un essere umano. Questi programmi funzionano o come utenti stessi delle \emph{chat} o come persone che rispondono alle FAQ delle persone che accedono al sito}
}

\newglossaryentry{CMS}
{
    name=\glslink{CMS}{Content management system},
    text=CMS,
    description={in informatica un \emph{content management system}, in acronimo CMS (sistema di gestione dei contenuti in italiano), è uno strumento software, installato su un server web, il cui compito è facilitare la gestione dei contenuti di siti web, svincolando il \emph{webmaster} da conoscenze tecniche specifiche di programmazione Web}
}

\newglossaryentry{diagramma di gantt}
{
    name=\glslink{diagramma di gantt}{Diagramma di Gantt},
    text=diagramma di Gantt,
    description={il diagramma di Gantt è uno strumento di supporto alla
gestione dei progetti, così chiamato in ricordo dell'ingegnere statunitense Henry
Laurence Gantt (1861-1919), che si occupava di scienze sociali e che lo ideò
nel 1917. Tale diagramma è usato principalmente nelle attività di \emph{project
management}, ed è costruito partendo da un asse orizzontale - a rappresentazione
dell’arco temporale totale del progetto, suddiviso in fasi incrementali (ad esempio,
giorni, settimane, mesi) - e da un asse verticale - a rappresentazione delle mansioni
o attività che costituiscono il progetto}
}

\newglossaryentry{XML}
{
    name=\glslink{XML}{Extensible Markup Language},
    text=XML,
    description={In informatica XML (sigla di \emph{eXtensible Markup Language}) è un metalinguaggio per la definizione di linguaggi di \emph{markup}, ovvero un linguaggio marcatore basato su un meccanismo sintattico che consente di definire e controllare il significato degli elementi contenuti in un documento o in un testo}
}

\newglossaryentry{FTP}
{
    name=\glslink{FTP}{File Transfer Protocol},
    text=FTP,
    description={in informatica il \emph{File Transfer Protocol} è un protocollo per la trasmissione di dati tra \emph{host} basato su \emph{TCP} e con architettura di tipo client-server. Il protocollo usa connessioni \emph{TCP} distinte per trasferire i dati e per controllare i trasferimenti e richiede autenticazione del client tramite nome utente e password}
}

\newglossaryentry{framework}
{
    name=\glslink{framework}{Framework},
    text=framework,
    description={un \emph{framework}, termine della lingua inglese che può essere tradotto come intelaiatura o struttura, in informatica e specificatamente nello sviluppo software, è un'architettura logica di supporto (spesso un'implementazione logica di un particolare \emph{design pattern}) su cui un software può essere progettato e realizzato, spesso facilitandone lo sviluppo da parte del programmatore},
    plural=frameworks
}

\newglossaryentry{FAQ}
{
    name=\glslink{FAQ}{Frequently Asked Questions},
    text=FAQ,
    description={Le \emph{Frequently Asked Questions}, meglio conosciute con la sigla FAQ, sono letteralmente le "domande poste frequentemente"; più esattamente sono una serie di risposte stilate direttamente dall'autore, in risposta alle domande che gli vengono poste, o che ritiene gli verrebbero poste, più frequentemente dagli utilizzatori di un certo servizio: soprattutto su Internet e in particolare nel web e nelle comunità virtuali vi sono domande ricorrenti alle quali si preferisce rispondere pubblicamente con un documento affinché non vengano poste troppo spesso, in modo da sciogliere i dubbi dei nuovi utenti}
}

\newglossaryentry{frontend}
{
    name=\glslink{frontend}{Frontend},
    text=frontend,
    description={termine che denota l’insieme delle applicazioni, relative ad una piattaforma
web, con le quali l’utente interagisce direttamente essendo la parte visibile da chiunque e raggiungibile all'indirizzo web del sito.}
}

\newglossaryentry{IDE}
{
    name=\glslink{IDE}{Integrated Development Environment},
    text=IDE,
    description={in informatica un ambiente di sviluppo
integrato (in lingua inglese \emph{integrated development environment} ovvero IDE,
anche integrated design environment o integrated debugging environment, rispettivamente
ambiente integrato di progettazione e ambiente integrato di \emph{debugging})
è un software che, in fase di programmazione, aiuta i programmatori nello sviluppo
del codice sorgente di un programma. Spesso l’IDE aiuta lo sviluppatore
segnalando errori di sintassi del codice direttamente in fase di scrittura, oltre a
tutta una serie di strumenti e funzionalità di supporto alla fase di sviluppo e
\emph{debugging}}
}

\newglossaryentry{JSON}
{
    name=\glslink{JSON}{JavaScript Object Notation},
    text=JSON,
    description={in informatica, nell’ambito della programmazione
web, JSON, acronimo di \emph{JavaScript Object Notation}, è un formato adatto
all’interscambio di dati fra applicazioni \emph{client-server}}
}

\newglossaryentry{NLP}
{
    name=\glslink{NLP}{Natural Language Processing},
    text=NLP,
    sort=nlp,
    description={in informatica con il termine \emph{Application Programming Interface API} (ing. interfaccia di programmazione di un'applicazione) si indica ogni insieme di procedure disponibili al programmatore, di solito raggruppate a formare un set di strumenti specifici per l'espletamento di un determinato compito all'interno di un certo programma. La finalità è ottenere un'astrazione, di solito tra l'hardware e il programmatore o tra software a basso e quello ad alto livello semplificando così il lavoro di programmazione}
}

\newglossaryentry{ORM}
{
    name=\glslink{ORM}{Object-relational mapping},
    text=ORM,
    description={In informatica l'\emph{Object-Relational Mapping} (ORM) è una tecnica di programmazione che favorisce l'integrazione di sistemi software aderenti al paradigma della programmazione orientata agli oggetti con sistemi RDBMS.
Un prodotto ORM fornisce, mediante un'interfaccia orientata agli oggetti, tutti i servizi inerenti alla persistenza dei dati, astraendo nel contempo le caratteristiche implementative dello specifico RDBMS utilizzato}
}

\newglossaryentry{open source}
{
    name=\glslink{open source}{Open source},
    text=open source,
    description={In informatica, il termine inglese \emph{open source} (che significa sorgente aperta) indica un software di cui gli autori (più precisamente, i detentori dei diritti) rendono pubblico il codice sorgente, favorendone il libero studio e permettendo a programmatori indipendenti di apportarvi modifiche ed estensioni. Questa possibilità è regolata tramite l'applicazione di apposite licenze d'uso. Il fenomeno ha tratto grande beneficio da Internet, perché esso permette a programmatori distanti di coordinarsi e lavorare allo stesso progetto}
}

\newglossaryentry{push}
{
    name=\glslink{push}{Notifica push},
    text=push,
    description={La notifica push è una tipologia di messaggistica istantanea con la quale il messaggio perviene al destinatario senza che questo debba effettuare un'operazione di scaricamento (modalità \emph{pull}). Tale modalità è quella tipicamente utilizzata da applicazioni come \emph{WhatsApp}, oppure servizi di sistemi operativi come \emph{Android}, o come numerose applicazioni derivate da siti web (ad esempio il classico servizio meteo o quello delle notizie).
Per permettere alle notifiche push di giungere al destinatario è indispensabile che l'applicazione sia attiva, ovvero operi in \emph{background} e sia on-line. Successivamente, occorre che l'utente abbia autorizzato l'applicazione a inviare le notifiche}
}

\newglossaryentry{Mercurial}
{
    name=\glslink{Mercurial}{Mercurial},
    text=Mercurial,
    description={\emph{Mercurial} è un software multipiattaforma di controllo di versione distribuito creato da Matt Mackall e disponibile sotto \emph{GNU General Public License 2.0}.
È quasi completamente scritto in \emph{Python}, ma include anche una implementazione \emph{diff} binaria scritta in C. Il programma ha un'interfaccia a riga di comando, ma incorpora anche un'elementare interfaccia web. 
Tutti i comandi di \emph{Mercurial} sono invocati come opzioni del programma principale hg, un riferimento al simbolo chimico dell'elemento mercurio}
}

\newglossaryentry{POST}
{
    name=\glslink{POST}{POST},
    text=POST,
    description={Il metodo POST è un metodo HTTP usato di norma per inviare informazioni
al server (ad esempio i dati di un \emph{form}). In questo caso l’URI indica che cosa si
sta inviando e il \emph{body} ne indica il contenuto}
}

\newglossaryentry{SaaS}
{
    name=\glslink{SaaS}{Software as a Service},
    text=SaaS,
    description={\emph{Software as a service} (SaaS) (Software come servizio in italiano) è un modello di distribuzione del \emph{software} applicativo dove un produttore di \emph{software} sviluppa, opera (direttamente o tramite terze parti) e gestisce un'applicazione web che mette a disposizione dei propri clienti via Internet previo abbonamento. Si tratta spesso, ma non sempre, di un servizio di \emph{cloud computing}}
}

\newglossaryentry{SDK}
{
    name=\glslink{SDK}{Software Development Kit},
    text=SDK,
    description={Un \emph{software development kit} (SDK, traducibile in italiano come "pacchetto di sviluppo per applicazioni"), in informatica, indica genericamente un insieme di strumenti per lo sviluppo e la documentazione di software. Molti SDK sono disponibili gratuitamente e possono essere prelevati direttamente dal sito del produttore: in questo modo si cerca di invogliare i programmatori ad utilizzare un determinato linguaggio o sistema}
}

\newglossaryentry{stakeholder}
{
    name=\glslink{stakeholder}{Stakeholder},
    text=stakeholder,
    description={in economia con il termine \emph{stakeholder} (o portatore di interesse)
si indica genericamente un soggetto (o un gruppo di soggetti) influente nei
confronti di un’iniziativa economica, che sia un’azienda o un progetto}
}

\newglossaryentry{SQL}
{
    name=\glslink{SQL}{Structured Query Language},
    text=SQL,
    description={In informatica SQL (Structured Query Language) è un linguaggio standardizzato per database basati sul modello relazionale progettato per:
\begin{itemize}
	\item creare e modificare schemi di database (DDL - Data Definition Language);
	\item inserire, modificare e gestire dati memorizzati (DML - Data Manipulation
Language);
	\item interrogare i dati memorizzati (DQL - Data Query Language);
	\item creare e gestire strumenti di controllo ed accesso ai dati (DCL - Data Control
Language)
\end{itemize}}
}

 % database di termini
\makeglossaries


%**************************************************************
% Impostazioni di graphicx
%**************************************************************
\graphicspath{{immagini/}} % cartella dove sono riposte le immagini


%**************************************************************
% Impostazioni di hyperref
%**************************************************************
\hypersetup{
    %hyperfootnotes=false,
    %pdfpagelabels,
    %draft,	% = elimina tutti i link (utile per stampe in bianco e nero)
    colorlinks=true,
    linktocpage=true,
    pdfstartpage=1,
    pdfstartview=FitV,
    % decommenta la riga seguente per avere link in nero (per esempio per la stampa in bianco e nero)
    %colorlinks=false, linktocpage=false, pdfborder={0 0 0}, pdfstartpage=1, pdfstartview=FitV,
    breaklinks=true,
    pdfpagemode=UseNone,
    pageanchor=true,
    pdfpagemode=UseOutlines,
    plainpages=false,
    bookmarksnumbered,
    bookmarksopen=true,
    bookmarksopenlevel=1,
    hypertexnames=true,
    pdfhighlight=/O,
    %nesting=true,
    %frenchlinks,
    urlcolor=webbrown,
    linkcolor=RoyalBlue,
    citecolor=webgreen,
    %pagecolor=RoyalBlue,
    %urlcolor=Black, linkcolor=Black, citecolor=Black, %pagecolor=Black,
    pdftitle={\myTitle},
    pdfauthor={\textcopyright\ \myName, \myUni, \myFaculty},
    pdfsubject={},
    pdfkeywords={},
    pdfcreator={pdfLaTeX},
    pdfproducer={LaTeX}
}

%**************************************************************
% Impostazioni di itemize
%**************************************************************
\renewcommand{\labelitemi}{$\ast$}

%\renewcommand{\labelitemi}{$\bullet$}
%\renewcommand{\labelitemii}{$\cdot$}
%\renewcommand{\labelitemiii}{$\diamond$}
%\renewcommand{\labelitemiv}{$\ast$}


%**************************************************************
% Impostazioni di listings
%**************************************************************
\lstset{
    language=[LaTeX]Tex,%C++,
    keywordstyle=\color{RoyalBlue}, %\bfseries,
    basicstyle=\small\ttfamily,
    %identifierstyle=\color{NavyBlue},
    commentstyle=\color{Green}\ttfamily,
    stringstyle=\rmfamily,
    numbers=none, %left,%
    numberstyle=\scriptsize, %\tiny
    stepnumber=5,
    numbersep=8pt,
    showstringspaces=false,
    breaklines=true,
    frameround=ftff,
    frame=single
} 


%**************************************************************
% Impostazioni di xcolor
%**************************************************************
\definecolor{webgreen}{rgb}{0,.5,0}
\definecolor{webbrown}{rgb}{.6,0,0}


%**************************************************************
% Altro
%**************************************************************

\newcommand{\omissis}{[\dots\negthinspace]} % produce [...]

% eccezioni all'algoritmo di sillabazione
\hyphenation
{
    ma-cro-istru-zio-ne
    gi-ral-din
}

\newcommand{\sectionname}{sezione}
\addto\captionsitalian{\renewcommand{\figurename}{Figura}
                       \renewcommand{\tablename}{Tabella}}

\newcommand{\glsfirstoccur}{\ap{{[g]}}}

\newcommand{\intro}[1]{\emph{\textsf{#1}}}

%**************************************************************
% Environment per ``rischi''
%**************************************************************
\newcounter{riskcounter}                % define a counter
\setcounter{riskcounter}{0}             % set the counter to some initial value

%%%% Parameters
% #1: Title
\newenvironment{risk}[1]{
    \refstepcounter{riskcounter}        % increment counter
    \par \noindent                      % start new paragraph
    \textbf{\arabic{riskcounter}. #1}   % display the title before the 
                                        % content of the environment is displayed 
}{
    \par\medskip
}

\newcommand{\riskname}{Rischio}

\newcommand{\riskdescription}[1]{\textbf{\\Descrizione:} #1.}

\newcommand{\risksolution}[1]{\textbf{\\Soluzione:} #1.}

%**************************************************************
% Environment per ``use case''
%**************************************************************
\newcounter{usecasecounter}             % define a counter
\setcounter{usecasecounter}{0}          % set the counter to some initial value

%%%% Parameters
% #1: ID
% #2: Nome
\newenvironment{usecase}[2]{
    \renewcommand{\theusecasecounter}{\usecasename #1}  % this is where the display of 
                                                        % the counter is overwritten/modified
    \refstepcounter{usecasecounter}             % increment counter
    \vspace{10pt}
    \par \noindent                              % start new paragraph
    {\large \textbf{\usecasename #1: #2}}       % display the title before the 
                                                % content of the environment is displayed 
    \medskip
}{
    \medskip
}

\newcommand{\usecasename}{UC}

\newcommand{\usecaseactors}[1]{\textbf{\\Attori Principali:} #1. \vspace{4pt}}
\newcommand{\usecasepre}[1]{\textbf{\\Precondizioni:} #1. \vspace{4pt}}
\newcommand{\usecasedesc}[1]{\textbf{\\Descrizione:} #1. \vspace{4pt}}
\newcommand{\usecasepost}[1]{\textbf{\\Postcondizioni:} #1. \vspace{4pt}}
\newcommand{\usecasealt}[1]{\textbf{\\Scenario Alternativo:} #1. \vspace{4pt}}

%**************************************************************
% Environment per ``namespace description''
%**************************************************************

\newenvironment{namespacedesc}{
    \vspace{10pt}
    \par \noindent                              % start new paragraph
    \begin{description} 
}{
    \end{description}
    \medskip
}

\newcommand{\classdesc}[2]{\item[\textbf{#1:}] #2}